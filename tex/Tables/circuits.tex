\begin{table}[htb]\footnotesize
  \centering
  \begin{minipage}[b]{.6\linewidth}
    \centering
    \begin{tabular}{|c|c|c|c|}
      \hline
      $n$  & Inputs & Coeffs   & Muls    \\\hline
      10   & 13     & 24       & 128     \\\hline
      20   & 25     & 53       & 545     \\\hline  
      50   & 66     & 129      & 3472    \\\hline  
      100  & 127    & 307      & 16178   \\\hline  
    \end{tabular}
    %
    \begin{tabular}{|c|c|c|c|c|}
      \hline
      $n$  & Inputs & Coeffs   & Muls    \\\hline
      125  & 149    & 312      & 19766   \\\hline  
      150  & 186    & 378      & 29075   \\\hline  
      200  & 269    & 532      & 55473   \\\hline  
      300  & 372    & 741      & 112195  \\\hline  
    \end{tabular}
    \subcaption{Average dimensions of the circuits computing our randomly
    generated polynomials of degree $n$.}
  \end{minipage}%
  \begin{minipage}[b]{.4\linewidth}
    \centering
    \begin{tabular}{|c|c|c|c|}
      \hline
      $n$ & Inputs & Coeffs   & Muls    \\\hline
      12  & 4      & 247      & 186     \\\hline  
      30  & 6      & 16807    & 55250   \\\hline  
      35  & 6      & 32768    & 110624  \\\hline  
      40  & 6      & 131022   & 203428  \\\hline  
      50  & 6      & 500040   & 350042  \\\hline  
    \end{tabular}%
    \subcaption{Dimensions of the example circuits of degree $n$ included with
      \citetool{parno:howell:gentry:raykova:2013}.}
  \end{minipage}%
  \captionof{table}[Arithmetic circuits' dimensions]{Dimensions of our randomly
    generated circuits and from \citetool{parno:howell:gentry:raykova:2013}'s
    polynomial examples. \citetool{parno:howell:gentry:raykova:2013}'s
    polynomials are of lower degree, but have much more coefficients and
    multiplications than ours.}\label{tab:circuits}
  %\captionof{table}[Arithmetic circuits' average dimensions.]{Average number of
  %inputs and product gates of the randomly generated circuits, where for
  %each polynomial of degree $n$ we have 100 circuits.}\label{tab:circuits} 
\end{table}
