\pgfplotstableread{Data/polymul-128.dat}{\polymulone}
\pgfplotstableread{Data/polymul-256.dat}{\polymultwo}

\begin{figure}
  \centering
  \begin{minipage}[b]{.5\linewidth}
    %%%%%%%%%%%%%%%%%%%%%%
    \begin{tikzpicture}[scale=0.9]
      \begin{axis}[
        xlabel={$n$},
        every axis x label/.style={
          at={(axis description cs:0.5,-0.05)},
          anchor=north,
        },
        ylabel={Time (ms)},
        every axis y label/.style={
          at={(axis description cs:-0.2,0.5)},
          rotate=90,
          anchor=north,
        },
        legend style={at={(0.4,0.9)}},
        axis x line=bottom,
        axis y line=left,
        style={font=\scriptsize},
        ]
        
        \addplot+[smooth,mark=none,red,line width=0.5pt] table
          [x={n}, y={classical},
          %select coords between index={1}{10}
          ]
            {\polymulone};
        \addlegendentry{$\lambda = 128$};
        
        \addplot+[smooth,mark=none,blue,line width=0.5pt] table
          [x={n}, y={classical},
          %select coords between index={1}{10}
          ]
            {\polymultwo};
        \addlegendentry{$\lambda = 256$};
      \end{axis}
    \end{tikzpicture}
    \subcaption{Classical $\Theta(n^2)$ multiplication.}\label{fig:polymul-nofft}
  \end{minipage}%
  \begin{minipage}[b]{.5\linewidth}
    %%%%%%%%%%%%%%%%%%%%%%
    \begin{tikzpicture}[scale=0.9]
      \begin{axis}[
        xlabel={$n$},
        every axis x label/.style={
          at={(axis description cs:0.5,-0.05)},
          anchor=north,
        },
        ylabel={Time (ms)},
        every axis y label/.style={
          at={(axis description cs:-0.15,0.5)},
          rotate=90,
          anchor=north,
        },
        legend style={at={(0.4,0.9)}},
        ytick={0,5,...,100},
        axis x line=bottom,
        axis y line=left,
        style={font=\scriptsize},
        ]
        
        \addplot+[smooth,mark=none,red,line width=0.5pt] table
          [x={n}, y={fft}
          %select coords between index={1}{10}
          ]
            {\polymulone};
        \addlegendentry{$\lambda = 128$};
        
        \addplot+[smooth,mark=none,blue,line width=0.5pt] table
          [x={n}, y={fft}
          %select coords between index={1}{10}
          ]
            {\polymultwo};
        \addlegendentry{$\lambda = 256$};
      \end{axis}
    \end{tikzpicture}
    \subcaption{$\Theta(n \log{n})$ multiplication with FFT.}\label{fig:polymul-fft}
  \end{minipage}%
  \caption{Times (in ms) of multiplication of two polynomials with degree $n
  = 25$ to $2500$. There is clearly a huge gain by using FFT multiplication.}
  \label{fig:polymul}
\end{figure}
