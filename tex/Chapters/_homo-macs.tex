%% ---------------------------------------------------------------------------
% HOMOMORPHIC MACS
%% ---------------------------------------------------------------------------
\section{Homomorphic MACs}\label{subsec:def-hmac}
The symmetric analogue of homomorphic signatures are homomorphic MACs.
%
Basically, Alice has a secret key which is used to generate a tag $\sigma$ that
authenticates a message $m$ under label $\tau$. Given a labeled program $\calP
= (f, \tauSpace{1}{n})$ and a set of tags $\sigmaSpace{1}{n}$, anyone can
execute the homomorphic evaluation algorithm over $\calP(\sigmaSpace{1}{n})$ to
generate a short tag $\sigma'$ that authenticates $m' = \calP(\mSpace{1}{n})$
as the output of the $\calP$ executed over inputs labeled by $\tauSpace{1}{n}$,
respectively.

There are three main properties that a homomorphic MAC scheme is required to
satisfy.%
\begin{inparaenum}[(1)]
\item It must be \emph{secure}, i.e., an adversary that asks for tags
  authenticating some messages of his own choice should not be able to produce
  valid tags authenticating messages that are not obtained as the output of
  $\calP$.
\item It should be \emph{succinct}, i.e., the output of $\calP$ can be
  certified using much less communication that than of sending the original
  authenticated messages.
\item Lastly, it should be \emph{composable} meaning that tags authenticating
  previous outputs of $\calP$ should be usable as inputs to further
  authenticate new computations, i.e., the tag authenticating the output of
  $\calP$ can be used as one of the inputs of another labeled program $\calP'$.
\end{inparaenum}

However, unlike in homomorphic signatures, it is necessary to explain what it
really means to authenticate a message as the output of a labeled program.
%
In a scenario where many users share the secret key and authenticate various
data-items without keeping any local or joint state, it is necessary to specify
which data is being authenticated and which data the program $\calP$ should be
evaluated on.
%
With this in mind, \textcite{gennaro:wichs:2012} defined the concept of
\emph{labeled programs}\footnote{\citeauthor{gennaro:wichs:2012} defined
labeled programs for boolean circuits $\function{f}{\{0, 1\}^n}{\{0,1\}}$, but
here we will consider the case for arithmetic circuits
$\function{f}{\bbF^n}{\bbF}$ where $\bbF$ is some finite field, like $\bbZ_p$
for some prime $p$.}.

\paragraph*{Labeled Programs} \emph{Labels} are used to index some data.
A labeled program $\calP = (f, \tauSpace{1}{n})$ consists of
a circuit\footnote{Or function.} $\function{f}{\bbF^n}{\bbF}$  and a distinct
input label $\tau_i \in \{0,1\}^\lambda$ for each input wire $i \in
\{\interval{1}{n}\}$ of the circuit. This can be seen as way to give useful
names to the variables of a program, without knowing the input values -- as an
example, we can consider a program that outputs the average salaries of
a company, and each $\tau_i$ can be of the form ``salary of worker $i$''.
%
A message $m$ is then authenticated with respect to a label $\tau$, and the
value of the label does not need to have any meaningful semantics. This
label-message association basically means that the value $m$ can be assigned to
those input variables of a labeled program $\calP$ whose label is $\tau$.
However, with this definition, a label \emph{cannot} be re-used for multiple
messages, i.e., two distinct messages $m, m'$ cannot be authenticated with
respect to the same label $\tau$. 

Given some labeled programs $\setSpace{\calP}{1}{t}$ and a circuit
$\function{g}{\bbF^t}{\bbF}$, the \emph{composed program} $\calP^*
= g(\setSpace{\calP}{1}{t})$ consists in evaluating a circuit $g$ on the
outputs of $\setSpace{\calP}{1}{t}$. The labeled inputs of the composed program
$\calP^*$ are just all the \emph{distinct} labeled inputs of
$\setSpace{\calP}{1}{t}$, and all the input wires with the same label are put
together in a single input wire.  The \emph{identity program} with label $\tau$
is denoted by $\calI_\tau := (g_{\id}, \tau)$, where $g_{\id}$ is the
\emph{canonical identity circuit} and $\tau \in \{0, 1\}^\lambda$ is some input
label. Any program $\calP = (f, \tauSpace{1}{n})$ can be written as
a composition of programs $\calP = f (\setSpace{\calI}{\tau_1}{\tau_n})$.

Given a labeled program $\calP = (f, \tauSpace{1}{n})$ and a set of tags
$\sigmaSpace{1}{n}$ that authenticates messages $m_i$ under label $\tau_i$
, anyone can run the homomorphic evaluation algorithm that takes an input the
tuple $(\calP, \sigmaSpace{1}{n})$ and whose output $\sigma'$ will authenticate
$m'$ as the output of $\calP(\mSpace{1}{n})$\footnote{To be more precise,
$\sigma'$ only certifies $m'$ as the output of a specific program $\calP$.}.
Then the secret-key verification algorithm takes as input the triple $(m',
\calP, \sigma')$ and verifies that $m'$ is indeed the output of the program
$\calP$ executed over some previously authenticated and labeled data, without
knowing the value of the original data.

\begin{definition}\label{def:hmac}
  A \emph{homomorphic message authenticator scheme} $\HMAC$ is a tuple of PPT
  algorithms (\KeyGen, \Auth, \Vrfy, \Eval), where $\calM$ defines a message
  space, as follows:
  \begin{description}
    \item[$\KeyGen(1^\lambda) \to (\ek, \sk)$:] Takes a security
      parameter $\lambda$. Outputs a public evaluation key \ek and a secret key
      \sk.
    \item[$\Auth_\sk(\tau, m) \to \sigma$:] Receives an input-label
      $\tau \in \{0, 1\}^\lambda$ and a message $m \in \calM$, and outputs a tag
      $\sigma$.
    \item[$\Vrfy_\sk(m, \calP, \sigma) \to \{\mathsf{accept},
      \mathsf{reject}\}$:] Receives a message $m \in \calM$, a labeled program
      $\calP = (f, \discretionary{}{}{}\tauSpace{1}{n})$ and a tag $\sigma$,
      and outputs either \textsf{accept} or \textsf{reject}.
    \item[$\Eval_\ek(f, \vec{\sigma}) \to \sigma'$: ] Receives
      a circuit $\function{f}{\calM^n}{\calM}$ and a vector of tags
      $\vec{\sigma} = (\sigmaSpace{1}{n})$ , and outputs a new tag $\sigma'$.
  \end{description}
\end{definition}

A homomorphic MAC scheme must achieve the following properties: authentication
and evaluation correctness, succinctness and security.

\paragraph*{Authentication correctness} For any $m \in \calM$, all keys $(\ek,
\sk) \gets_\$ \KeyGen(1^\lambda)$, any label $\tau \in \{0,1\}^\lambda$, and
any tag $\sigma \gets_\$ \Auth_\sk(\tau, m)$, it holds:
\begin{equation}\label{eq:hmac-auth-corr}
  \Pr[\Vrfy_\sk(m, \calI_\tau, \sigma) = \mathsf{accept}]
  = 1.
\end{equation}

\paragraph*{Evaluation correctness} For a pair $(\ek, \sk) \gets_\$
\KeyGen(1^\lambda)$, a circuit $\function{g}{\calM^t}{\calM}$ and any set of
triples $\{\left( m_i, \calP_i, \sigma_i \right)\}_{i = 1}^t$ such that
$\Vrfy_\sk(m_i, \calP_i, \sigma_i) = \mathsf{accept}$. If $m^*
= g(\mSpace{1}{t})$, $\calP^* = g(\setSpace{\calP}{1}{t})$, and $\sigma^*
= \Eval_\ek(g, (\sigmaSpace{1}{t}))$, then it must hold:
\begin{equation}\label{eq:hmac-eval-corr}
  \Vrfy_\sk(m^*, \calP^*, \sigma^*) = \mathsf{accept}.
\end{equation}

\paragraph*{Succinctness} The size of a computed tag is bounded by some fixed
polynomial in the security parameter $\poly(\lambda)$ which is independent of
the number $n$ of inputs taken by the evaluated circuit.

\paragraph*{Security} A homomorphic MAC scheme has to satisfy the following
notion of unforgeability.
\begin{definition}
  A homomorphic MAC scheme $\HMAC = (\KeyGen, \Auth, \Vrfy, \Eval)$ is
  \emph{unforgeable} if the advantage of any PPT adversary $\calA$ in the
  following game is negligible in the security parameter $\lambda$.
  \begin{description}
    \item[Setup] The challenger generates $(\ek, \sk) \gets_\$
      \KeyGen(1^\lambda)$ and gives the evaluation key \ek to $\calA$. It also
      initializes a list $Q = \emptyset$.
    \item[Authentication queries] The adversary can adaptively ask for tags on
      label-message pairs of her choice. Given a query $(\tau, m)$, if there is
      some $(\tau, m') \in Q$ for some $m' \neq m$, then the challenger ignores
      the query.
      %
      Otherwise, it computes $\sigma \gets \Auth_\sk(\tau, m)$, returns
      $\sigma$ to $\calA$ and updates the list $Q = Q \cup (\tau, m)$. If
      $(\tau, m) \in Q$ (i.e., the query was previously made), then the
      challenger returns the same tag generated before.
    \item[Verification queries] The adversary has access to a verification
      %oracle. $\calA$ can submit a query $(m, \calP, \sigma)$ and the
      oracle. $\calA$ can submit a query $(\tuple{m, \calP, \sigma})$ and the
      challenger replies with the output of $\Vrfy_\sk(m, \calP, \sigma)$.
    \item[Forgery] Eventually, the adversary outputs a forgery $(m^*, \calP^*
      = (f^*, \setSpace{\tau^*}{1}{n}), \sigma^*)$. Notice that such tuple can
      also be returned by $\calA$ as a verification query $(m^*, \calP^*,
      \sigma^*)$.
  \end{description}
  % Queries are not saved in Q because Auth is probabilistic
  Before describing the output of this game, it is necessary to define the
  notion of a well-defined program with respect to a list $Q$.
  %
  This notion intends to capture formally, which tuples generated by the
  adversary $\calA$ should be considered as valid forgeries. Because we are
  dealing with a homomorphic primitive, it should be possible to differentiate
  genuine tags produced by \Eval from tags produced in another, possibly
  malicious, way.
  %
  \paragraph*{Well-defined program} A labeled program $\calP^* = (f^*,
  \setSpace{\tau^*}{1}{n})$ is \emph{well-defined with respect to $Q$} if
  either one of the following conditions is met:
  \begin{enumerate}
    \item For some $i \in \{\interval{0}{n}\}$, there is a tuple $(\tau_i^*,
      \cdot) \notin Q$ (i.e., $\calA$ never asked authentication queries with
      label $\tau_i^*$), and $f^*(\{m_j\}_{(\tau_j, m_j) \in Q} \cup
      \{\tilde{m}_j\}_{(\tau_j, \cdot) \notin Q})$ outputs the same value for
      all possible values of $\tilde{m}_j \in \calM$. This means that the
      inputs $\tilde{m}_j$ do not affect the behavior of $f^*$ \footnote{
        Equivalently, it states that $f^*(\{m_j\}_{(\tau_j, m_j) \in Q}
        \cup\{\tilde{m}_j\}_{(\tau_j, \cdot) \notin Q})$ is semantically
        equivalent to $f^*(\{m_j\}_{(\tau_j, m_j) \in Q})$.}.

    %\item For $i = \{0, \dotsc, n\}$, all tuples $(\tau_i^*, m_i)$ are in $Q$.
    \item $Q$ contains tuples $(\tau_i^*, m_i)$ for some messages
      $\mSpace{1}{n}$, i.e., the entire input space of $f^*$ has been
      authenticated.
  \end{enumerate}
  The adversary $\calA$ wins if $\Vrfy_\sk(m^*, \calP^*, \sigma^*)
  = \mathsf{accept}$ and either:
  \begin{itemize}
    \item \emph{Type I forgery:} $\calP^*$ is not well-defined on $Q$ or,
    \item \emph{Type II forgery:} $\calP^*$ is well-defined on $Q$ and $m^*
      \neq f^*(\{m_j\}_{(\tau_j, m_j) \in Q})$, which means that $m^*$ is not
      the correct output of the labeled program $\calP^*$ when executed on
      previously authenticated messages $(\mSpace{1}{n})$.
  \end{itemize}
\end{definition}

%The aim of defining the notion of a well-defined program is to capture,
%formally, which tuples generated by the adversary $\calA$ should be considered
%as forgeries.
%
%But since we are dealing with a homomorphic primitive, we
%should be able to differentiate genuine tags produced by \Eval from tags
%produced in another, possibly malicious, way.
Notice, however, that even maliciously generated tags should not necessarily be
considered as forgeries.  This is because the adversary $\calA$ can trivially
modify a circuit $C$ she is allowed to evaluate, by adding dummy gates and
inputs that are simply ignored in the evaluation of the modified circuit. This
does not constitute an infringement of the security requirements because the
notion of a well-defined program $\calP$ captures this exact case: either
$\calP$ is run on legal inputs (i.e., inputs in $Q$) only, or, if this is not
the case, those inputs not queried (i.e., the dummy inputs not in $Q$) do not
affect the computation in any way.

The previous security definition is very similar to the one from fully
homomorphic MACs of \textcite{gennaro:wichs:2012}, except for two
modifications.  Here it is explicitly allowed to the adversary to query the
verification oracle, and forgeries are slightly different.
In~\cite{gennaro:wichs:2012}, Type I forgeries are defined as ones where at
least one new label is present, and Type II forgeries contain only labels that
have been queried, but $m^*$ is not the correct output of $\calP$ when executed
over previously authenticated messages.

For arbitrary computations, there is no efficient way to check whether
a program is well-defined with respect to a list $Q$. The main problem is to
check the first condition, i.e., whether a program always outputs the same
value for all possible choices of inputs that were not queried. However, for
the case of arithmetic circuits defined over the finite field $\bbZ_p$, where
$p$ is a prime of roughly $\lambda$ bits, and whose degree is bounded by
a polynomial, this check can be efficiently performed. In a follow-up work by
\textcite{generalhmac}, it is noted that testing whether a program is
well-defined can be done for arithmetic circuits of degree $d$, over a finite
field $\bbF$ of order $p$ such that $\frac{d}{p} < \frac{1}{2}$.

As one can observe from the authentication queries phase of the previous game,
it is explicitly not allowed to re-use a label to authenticate more than one
value. This is basically a way to keep track of the authenticated inputs. This
is a restriction that has been present on all previous works on homomorphic
authentication primitives.

\textcite{backes:fiore:reischuk:2013} extended the notion of labeled programs
in order to solve the problem of label re-use. Their notion of
\emph{multi-labeled programs} allows the partial, but safe, re-use of labels.

\paragraph*{Multi-labeled Programs} A \emph{multi-labeled program}
$\calP_\Delta$ is a pair $(\calP, \Delta)$ where $\calP = f(\tauSpace{1}{n})$
is a labeled-program and $\Delta \in \{0, 1\}^\lambda$ is a binary string
denominated \emph{data set label}. Basically, the combination of a input label
$\tau_i$ and a data set label $\Delta$, defined as a multi-label $\mathsf{L}
= (\Delta, \tau_i)$, is used to uniquely identify a specific data item. In
particular, binding a message $m_i$ with a multi-label $\mathsf{L}
= (\Delta,\tau_i)$ means that $m_i$ can be assigned to those input variables
with input label $\tau_i$.  The multi-label \textsf{L} uniquely identifies the
message $m_i$. While the re-use of a multi-label \textsf{L} is not allowed, the
re-use of a input label $\tau_i$ is allowed, instead.
%
Composition of multi-labeled programs within the same data set is possible.
Given multi-labeled programs $(\calP_1, \Delta), \dotsc, (\calP_t, \Delta)$
having the same data set label $\Delta$, and given a function
$\function{g}{\calM^t}{\calM}$, the \emph{composed multi-labeled program}
$\calP^*_\Delta$ is the pair $(\calP^*, \Delta)$ where $\calP^*$ is the
composed program $g(\setSpace{\calP}{1}{t})$, and $\Delta$ is the data set
label shared by all $\calP_i$. If $\function{f_{\id}}{\calM}{\calM}$ is the
canonical identity function and $\mathsf{L} = (\Delta, \tau) \in (\{0,
1\}^\lambda)^2$ is a multi-label, then $\calI_\mathsf{L} = (f_{\id},
\mathsf{L})$ is the \emph{identity multi-labeled program} for data set $\Delta$
and input label $\tau$. Like in labeled programs, any multi-labeled program
$\calP_\Delta = ((f, \tauSpace{1}{n}), \Delta)$ can be expressed as the
composition of $n$ identity multi-labeled programs $\calP_\Delta
= (\setSpace{\calI}{\mathsf{L}_1}{\mathsf{L}_n})$ where $\mathsf{L}_i
= (\Delta, \tau_i)$.

Having defined the notion of a multi-labeled program, the definition of
a homomorphic message authenticator from \refdef{def:hmac} can be adapted to
support multi-labeled programs.

\begin{definition}
  A homomorphic message authenticator $\HMACML$ is a tuple of PPT algorithms
  (\KeyGen, \Auth, \Vrfy, \Eval), where $\calM$ defines a message space, as
  follows:
  \begin{description}
    \item[$\KeyGen(1^\lambda) \to (\ek, \sk)$:] Takes a security
      parameter $\lambda$. Outputs a public evaluation key \ek and a secret key
      \sk.
    \item[$\Auth_\sk(\mathsf{L}, m) \to \sigma$:] Receives
      a multi-label $\mathsf{L} = (\Delta, \tau) \in (\{0, 1\}^\lambda)^2$ and
      a message $m \in \calM$, and outputs a tag $\sigma$.
    \item[$\Vrfy_\sk(m, \calP_\Delta, \sigma) \to \{\mathsf{accept},
      \mathsf{reject}\}$:] Receives a message $m \in \calM$, a multi-labeled program
      $\calP_\Delta = ((f, \tauSpace{1}{n}), \Delta)$ and a tag $\sigma$, and
      outputs either \textsf{accept} or \textsf{reject}.
    \item[$\Eval_\ek(f, \vec{\sigma}) \to \sigma'$:] Receives a circuit
      $\function{f}{\calM^n}{\calM}$ and a vector of tags $\vec{\sigma}
      = (\sigmaSpace{1}{n})$, and outputs a new tag $\sigma'$.
  \end{description}
\end{definition}

\paragraph*{Authentication correctness} For any message $m \in \calM$, all keys
$(\ek, \sk) \gets_\$ \KeyGen(1^\lambda)$, any multi-label $\mathsf{L}
= (\Delta, \tau) \in (\{0,1\}^\lambda)^2$, and any tag $\sigma' \gets_\$
\Auth_\sk(\mathsf{L}, m)$, it holds:
\begin{equation}\label{eq:hmac-ml-auth-corr}
  \Pr[ \Vrfy_\sk(m, \calI_\mathsf{L}, \sigma') = \mathsf{accept}] = 1.
\end{equation}

\paragraph*{Evaluation correctness} For a pair $(\ek, \sk) \gets_\$
\KeyGen(1^\lambda)$, a circuit $\function{g}{\calM^t}{\calM}$ and any set of
triples $\{(m_i, \calP_{\Delta, i}, \sigma_i\}_{i=1}^t$ such that all
multi-label programs $\calP_{\Delta, i} = (\calP_i, \Delta)$ and
$\Vrfy_\sk(m_i, \calP_{\Delta,i}, \sigma_i) = \mathsf{accept}$. If $m^*
= g(\mSpace{1}{t})$, $\calP^* = g(\setSpace{\calP}{1}{t})$, and $\sigma^*
= \Eval_\ek(g, (\sigmaSpace{1}{t}))$, then it must hold:
\begin{equation}\label{eq:hmac-ml-eval-corr}
  \Pr[\Vrfy_\sk(m^*, \calP^*_\Delta, \sigma^*) = \mathsf{accept}] = 1.
\end{equation}

\paragraph*{Succinctness} The size of a computed tag is bounded by some fixed
polynomial in the security parameter $\poly(\lambda)$ which is independent of
the number $n$ of inputs taken by the evaluated circuit.

\paragraph*{Security} The security property is extended from
\textcite{catalano:fiore:2013} to the model of multi-labeled programs.
A homomorphic MAC scheme has to satisfy the following notion of unforgeability.
%
\begin{definition}
  A homomorphic MAC scheme $\HMAC = (\KeyGen, \Auth, \Vrfy, \Eval)$ is
  \emph{unforgeable} if the advantage of any PPT adversary $\calA$ in the
  following game is negligible in the security parameter $\lambda$.
  \begin{description}
    \item[Setup] The challenger generates $(\ek, \sk) \gets_\$
      \KeyGen(1^\lambda)$ and gives the evaluation key to $\calA$.
    \item[Authentication queries] The adversary can adaptively ask for tags on
      multi-labels and messages of her choice. Given a query $(\mathsf{L}, m)$
      where $\mathsf{L} = (\Delta, \tau)$, if this is the first query with data
      set $\Delta$, the challenger initializes a list $Q_\Delta = \emptyset$.
      Then, if $(\tau, m') \in T_\Delta$ for some $m' \neq m$, the challenger
      ignores the query.  Otherwise, if $(\tau, \cdot) \notin Q_\Delta$, the
      challenger computes $\sigma' \gets_\$ \Auth_\sk(\mathsf{L}, m)$,
      returns $\sigma'$ to $\calA$ and updates list $Q_\Delta = Q_\Delta \cup
      (\tau, m)$. If $(\tau, m) \in Q_\Delta$, then the challenger returns the
      same tag generated before.
    \item[Verification queries] The adversary has access to a verification
      oracle. $\calA$ can submit a query $(\tuple{m, \calP_\Delta, \sigma})$, and the
      challenger replies with the output of $\Vrfy_\sk(m, \calP_\Delta,
      \sigma)$.
    \item[Forgery] The game ends when $\calA$ returns a forgery $(m^*,
      \calP^*_{\Delta^*} = (\calP^*, \Delta^*), \sigma^*)$ for some $\calP^*
      = (f^*, \setSpace{\tau^*}{1}{n})$. Notice that such tuple can be returned
      by $\calA$ as a verification query $(m^*, \calP^*_{\Delta^*}, \sigma^*)$.
  \end{description}
  %
  Just as with labeled programs, it is necessary to define the notion of
  well-defined programs with respect to a list $Q_\Delta$. The notion is
  exactly the same as for labeled programs, except that now we are within
  a data set $\Delta$.
  
  \noindent The adversary $\calA$ wins the game if $\Vrfy_\sk(m^*, \calP^*_{\Delta^*},
  \sigma^*) = \mathsf{accept}$ and one of the following holds:
  \begin{itemize}
    \item \emph{Type I Forgery}: no list $Q_\Delta^*$ was created, i.e., no
      message $m$ has been authenticated within data set label $\Delta^*$.
    \item \emph{Type II Forgery}: $\calP^*$ is well-defined with respect to
      $Q_{\Delta^*}$ and $m^* \neq f^*(\{m_j\}_{(\tau_j, m_j) \in Q_{\Delta^*}})$,
      i.e., $m^*$ is not the correct output of labeled program $\calP^*$ when
      executed on previously authenticated message $(\mSpace{1}{n})$.
    \item \emph{Type III Forgery}: $\calP^*$ is not well-defined with respect
      to $Q_{\Delta^*}$.
  \end{itemize}
\end{definition}

This definition of security is similar to the one from
\textcite{catalano:fiore:2013} previously presented, but extended to the model
of multi-labeled programs.
%
Just as with labeled programs, it is not possible to check in polynomial time
if whether an arbitrary computation is well-defined with respect to a list $Q$,
but for the specific case of arithmetic circuits defined over the finite field
$\bbZ_p$ and of polynomial degree, this check can be efficiently performed.
% NOTE: it is possible to convert a type 3 into a type 2. This conversion is
% what makes the checking of a well-defined program efficient, right?

\paragraph*{Efficient Verification}The notion of multi-labels by itself does
not guarantee an efficient verification algorithm. To achieve this it is
necessary to introduce a property of efficient verification. This efficiency
property is defined in an amortized sense, meaning that the verification is
more efficient when the same program $\calP$ is executed on different data
sets.
%
To achieve efficient verification, the definition of a homomorphic MAC scheme
previously defined has to be augmented with two new algorithms.
\begin{definition}
  A homomorphic MAC scheme $\HMACML = (\KeyGen, \Auth, \Vrfy, \Eval)$ satisfies
  \emph{efficient verification} if there exist two additional algorithms
  (\VrfyPrep, \EffVrfy) as follows:
  \begin{description}
    \item[$\VrfyPrep_\sk(\calP) \to \vk_\calP$:] Given a labeled
      program $\calP = (f, \tauSpace{1}{n})$, it generates a \emph{concise}
      verification key $\vk_\calP$. This verification key does not depend on
      any data set label $\Delta$.
    \item[$\EffVrfy_{\sk, \vk_\calP}(\Delta, m, \sigma)$:] Given a data set
      label $\Delta$, a message $m \in \calM$ and a tag $\sigma$, it outputs
      either \textsf{accept} or \textsf{reject}.
  \end{description}
\end{definition}
These new algorithms need to achieve two properties: correctness and amortized
efficiency.

\paragraph*{Correctness} Let $(\ek, \sk) \gets_\$ \KeyGen(1^\lambda)$ be a pair
of honestly generated keys, and $(m, \calP_\Delta, \sigma)$ be any tuple
message/program/tag with $\calP_\Delta = (\calP, \Delta)$ such that
$\Vrfy_\sk(m, \calP_\Delta, \sigma) = \mathsf{accept}$. Then, for every
$\vk_\calP \gets_\$ \VrfyPrep_\sk(\calP)$, it holds that:
\begin{equation}\label{eq:eff-vrfy-corr}
  \Pr[\EffVrfy_{\sk, \vk_\calP}(\Delta, m, \sigma) = \mathsf{accept}] = 1.
\end{equation}

\paragraph*{Amortized efficiency} Let $\calP_\Delta = (\calP, \Delta)$ be
a multi-labeled program, let $(\mSpace{1}{n}) \in \calM^n$ be any vector of
messages, and let $t(n)$ be the time required to compute
$\calP(\mSpace{1}{n})$. If $\vk_\calP \gets \VrfyPrep_\sk(\calP)$, then the
time required for $\EffVrfy_{\sk, \vk_\calP}(\Delta, m, \sigma)$ is $\calO(1)$
(independent of $n$).

In this efficiency requirement the cost of computing $\vk_\calP$ is not
considered. That is because the same $\vk_\calP$ can be re-used in many
verification operations with the same labeled program $\calP$, but many
different data sets $\Delta$. Given this, the cost of computing $\vk_\calP$ is
\emph{amortized} over many verifications of the same function on different data
sets.

\begin{comment}
\subsection{Homomorphic Evaluation of Arithmetic Circuits} Here we
follow~\cite{backes:fiore:reischuk:2013} to present some basic definitions
regarding the homomorphic evaluation of arithmetic circuits over values defined
in some appropriate set $\calJ \neq \calM$. Usually,
$\function{f}{\bbZ_p^n}{\bbZ_p}$, but sometimes the message space may be
defined in $\calJ$, and so it is necessary to obtain an equivalent algorithm
that evaluates messages from $\calJ$ over $f$.

\paragraph*{Homomorphic Evaluation over Polynomials} Consider that
$\calJ_{\poly} = \bbZ_p[x_1, \dotsc, x_m]$ is the ring of polynomials in
variables $x_1, \dotsc, x_m$ over $\bbZ_p$. For every tuple $\vec{a} = (a_1,
\dotsc, a_m) \in \bbZ_p^m$, let
$\function{\phi_{\vec{a}}}{\calJ_{\poly}}{\bbZ_p}$ be the function defined as
$\phi_{\vec{a}}(y) = y(a_1, \dotsc, a_m)$ for any $y \in \calJ_{\poly}$.
$\phi_{\vec{a}}$ is a homomorphism from $\calJ_{\poly}$ to $\bbZ_p$.
%
Given an arithmetic circuit $\function{f}{\bbZ_p^n}{\bbZ_p}$, there exists
another structurally equivalent circuit
$\function{\hat{f}}{\calJ_{\poly}^n}{\calJ_{\poly}}$ such that $\forall y_1,
\dotsc, y_n \in \calJ_{\poly} \colon \phi_{\vec{a}}(\hat{f}(y_1, \dotsc, y_n))
= f(\phi_{\vec{a}}(y_1), \dotsc, \phi_{\vec{a}}(y_n))$. The only difference is
that in every gate the operation in $\bbZ_p$ is replaced by the corresponding
operation in $\bbZ_p[x_1, \dotsc, x_m]$.

More precisely, for every positive integer $m \in \bbN$ and a given $f$, the
computation of $\hat{f}$ on $(y_1, \dotsc, y_n) \in \calJ^n_{\poly}$ can be
defined as $\PolyEval(m, f, y_1, \dotsc, y_n) \to \bbZ_p$. For every gate
$f_g$, on input two polynomials $y_1, y_2 \in \calJ_{\poly}$, proceeds as
follows: if $f_g = +$, it outputs $y = y_1 + y_2$ (adds all coefficients
component-wise); if $f_g = \times$, it outputs $y = y_1 \cdot y_2$ (uses the
convolution operator on the coefficients). Notice that every $\times$ gate
increases the degree $d$ of $y$, as well as its number of coefficients. If
$y_1, y_2$ have degree $d_1, d_2$ respectively, the degree of $y = y_1 \cdot
y_2$ is $d_1 + d_2$.

\paragraph*{Bilinear Groups} Let $\calG(1^\lambda)$ be an algorithm that on
input the security parameter $1^\lambda$, outputs the description of a bilinear
group $\bgpp = (p, \bbG, \bbG_T, e, g)$ where $\bbG$ and $\bbG_T$ are groups of
the same order $p > 2^\lambda$, $g \in \bbG$ is a generator and
$\function{e}{\bbG \times \bbG}{\bbG_T}$ is an efficiently computable bilinear
map, or pairing function. $\calG$ is a \emph{bilinear group generator}.

\paragraph*{Homomorphic Evaluation over Bilinear Groups} Let $\bgpp = (p, \bbG,
\bbG_T, e, g)$ be the description of a bilinear group as previously defined.
With a fixed generator $g \in \bbG$, then $\bbG \cong (\bbZ_p, +)$ (i.e.,
$\bbG$ and the additive group $(\bbZ_p, +)$ are isomorphic) by considering the
isomorphism $\phi_g(x) = g^x$ for every $x \in \bbZ_p$. Similarly, by the
property of the pairing function $e$, $\bbG_T \cong (\bbZ_p, +)$ by considering
the isomorphism $\phi_{g_T}(x) = e(g,g)^x$. Since both $\phi_g$ and
$\phi_{g_T}$ are isomorphisms, there also exist the corresponding inverses
$\function{\phi_g^{-1}}{\bbG}{\bbZ_p}$ and
$\function{\phi_{g_T}^{-1}}{\bbG_T}{\bbZ_p}$, even though these are not known
to be efficiently computable.

For every arithmetic circuit $\function{f}{\bbZ_p^n}{\bbZ_p}$ of degree at most
2, the $\GroupEval(f, X_1, \dotsc, X_n)$ algorithm homomorphically evaluates
$f$ with inputs in $\bbG$ and outputs in $\bbG_T$.

Basically, given a circuit $f$ of degree at most 2, and given an $n$-tuple of
values $(X_1, \dotsc, X_n) \in \bbG^n$, \GroupEval proceeds gate-by-gate as
follows: if $f_g = +$, it uses the group operation in $\bbG$ or in $\bbG_T$; if
$f_g = \times$, it uses the pairing function, thus ``lifting'' the result to
the group $\bbG_T$. By using only circuits of degree at most 2, the
multiplication is well defined.

\GroupEval achieves the desired homomorphic property:
\begin{theorem}\label{theo:group-eval-homo-prop}
  Let $\bgpp = (p, \bbG, \bbG_T, e, g)$ be the description of bilinear groups.
  Then, the algorithm $\GroupEval$ satisfies: $\forall(X_1, \dotsc, X_n) \in
  \bbG^n \colon \GroupEval(f, X_1, \dotsc, X_n) = e(g,g)^{f(x_1, \dotsc, x_n)}$
  for the unique values $\{x_i\}^n_{i=1} \in \bbZ_p$ such that $X_i = g^{x_i}$.
\end{theorem}
The proof of \reftheorem{theo:group-eval-homo-prop} can be found
in~\cite{backes:fiore:reischuk:2013}.

\subsection{Pseudo-Random Functions with Amortized Closed-Form Efficiency}
A \emph{closed-form efficient} PRF is just like a standard PRF, plus an
additional efficiency property. Consider a computation $\Comp(R_1, \dotsc, R_n,
\vec{z})$ which takes random inputs $R$ and arbitrary inputs $\vec{z}$, and
runs in time $t(n, |\vec{z}|)$. By considering the case where each $R_i
= \mathsf{F}_K(\mathsf{L}_i)$, then the PRF \textsf{F} is said to satisfy
closed-form efficiency for $(\Comp, \vec{\mathsf{L}})$ if, with the knowledge
of the seed $K$, one can compute $\Comp(\mathsf{F}_K(\mathsf{L}_1), \dotsc,
\mathsf{F}_K(\mathsf{L}_n), \vec{z})$ in time strictly less than $t$. The main
idea is that in the pseudo-random case all $R_i$ values have a shorter
closed-form representation (as a function of $K$), and this might also allow
for the existence of a shorter closed-form representation of the computation
\Comp.

From the above considerations it possible to define a new property for PRFs:
\emph{amortized close-form efficiency}. The idea is to address computations
where each $R_i$ is generated as $\mathsf{F}_K(\Delta, \tau_i)$. All the inputs
of \textsf{F} share the same data set label $\Delta$. Then, \textsf{F}
satisfies amortized closed-form efficiency if it is possible to compute $\ell$
computations $\{\Comp(\mathsf{F}_K(\Delta_j, \tau_1), \dotsc,
\mathsf{F}_K(\Delta_j, \tau_n), \vec{z})\}^\ell_{j = 1}$ in time strictly less
than $\ell \cdot t$.

A PRF is defined as follows:
\begin{description}
  \item[$\KeyGen(1^\lambda) \to (\pp, K)$:] Given a security parameter,
    output a secret key $K$ and some public parameters $\pp$ that specify
    domain $\calX$ and range $\calR$ of the function.
  \item[$\F_K(x) \to R$:] Receives $x \in \calX$ and uses the secret
    key $K$ to compute a value $R \in \calR$.
\end{description}
A PRF must satisfy the pseudo-randomness property. More precisely, (\KeyGen, \F)
is \emph{secure} if for every PPT adversary $\calA$ it holds:
\begin{equation}\label{eq:amor-closed-form-prf-sec}
  \abs*{\Pr[\calA^{\F_K(\cdot)}(1^\lambda, \pp) = 1]
  - \Pr[\calA^{\phi(\cdot)}(1^\lambda, \pp) = 1]} \leq \negl(\lambda)
\end{equation}
$(K, \pp) \gets_\$ \KeyGen(1^\lambda)$, and
$\function{\phi}{\calX}{\calR}$ is a random function.

\begin{definition}\label{def:amort-closed-form-eff-prf}
  Consider a computation $\Comp(R_1, \dotsc, R_n, z_1, \dotsc, z_m)$ with
  random $n$ input values and $m$ arbitrary input values. \Comp takes time
  $t(n,m)$. Let $\vec{\mathsf{L}} = (\mathsf{L}_1, \dotsc, \mathsf{L}_n)$ be
  arbitrary values in the domain $\calX$ of \textsf{F} such that $\mathsf{L}_i
  = (\Delta, \tau_i)$. A PRF (\KeyGen, \F) satisfies \emph{amortized
  closed-form efficiency} for $(\Comp, \vec{\mathsf{L}})$ if there exist
  algorithms $\CFEval^\off_{\Comp, \vec{\tau}}$ and $\CFEval^\on_{\Comp,
  \Delta}$ such that:
  \begin{enumerate}
    \item Given $\omega \gets \CFEval^\off_{\Comp, \vec{\tau}}(K,
      \vec{z})$, then
      \begin{equation}\label{eq:prf-acf-eff-1}
        \CFEval^\on_{\Comp, \Delta}(K, \omega) = \Comp(\F_K(\Delta, \tau_1),
        \dotsc, \F_K(\Delta, \tau_n), z_1, \dotsc, z_m)
      \end{equation}
    \item The running time of $\CFEval^\on_{\Comp, \Delta}(K, \omega)$ is
      $o(t)$.
  \end{enumerate}
\end{definition}
The $\omega$ obtained from the offline computation $\CFEval^\off_{\Comp, \tau}$
does not depend on data set label $\Delta$, which means that it can be re-used
in further online computations $\CFEval^\on_{\Comp, \Delta}(K, \omega)$ to
compute $\Comp(\F_K(\Delta, \tau_1), \dotsc, \F_K(\Delta, \tau_n), \vec{z})$
for many different $\Delta$'s.
%
Because of this, the efficiency property puts a restriction only on
$\CFEval^\on_{\Comp,\Delta}$ in order to capture the idea of achieving
efficiency in an amortized sense when considering many evaluations over of
$\Comp(\F_K(\Delta, \tau_1), \dotsc, \F_K(\Delta, \tau_n), \dotsc, \vec{z})$,
with different data set label $\Delta$ in each evaluation. Essentially, one can
pre-compute $\omega$ once, and then use it to run the online phase as many
times as needed, almost for free.

The structure of \Comp may impose some restrictions on the range $\calR$ of the
PRF, and due to the pseudo-randomness property, the output distribution of
$\CFEval^\on_{\Comp, \Delta}(K, \CFEval^\off_{\Comp, \vec{\tau}}(K, \vec{z}))$
is computationally indistinguishable from the output distribution of
$\Comp(R_1, \dotsc, R_n, \vec{z})$.
\end{comment}

%% ---------------------------------------------------------------------------
% CATALANO-FIORE CONSTRUCTION
%% ---------------------------------------------------------------------------
\subsection{Catalano-Fiore}These constructions only support a restricted set of functions. More precisely,
they support polynomials $\{f_n\}$ over $\bbF$ which have \emph{polynomially
bounded degree}, meaning that both the number of variables and the degree of
$f_n$ are bounded by some polynomial $p(n)$. The class $\mathcal{VP}$ contains
all polynomially bounded degree families of polynomials that are defined by
arithmetic circuits of polynomial size and degree.

\subsubsection*{Homomorphic MAC from OWFs}
The security of this construction relies only on a \nom{PRF}{Pseudo-Random
Function} (and thus on \nom{OWF}{One-Way Function}s). It is very simple and
efficient, and allows the homomorphic evaluation of circuits
$\function{f}{\bbZ^n_p}{\bbZ_p}$ for a prime $p$ of roughly $\lambda$ bits.

This construction is restricted to circuits whose additive gates do not get
inputs labeled by constants. Without loss of generality, and when needed, one
can use an equivalent circuit where there is a special variable/label for the
value of 1, and the MAC of 1 can be published.

The description of the scheme (which we will refer to as
\citescheme{catalano:fiore:2013}1) with security parameter $\lambda$ is as
follows:
\begin{description}
  \item[$\KeyGen(1^\lambda) \to (\ek, \sk)$.] Let $p$ be a prime of
    roughly $\lambda$ bits. Choose a seed $K \gets_\$ \{0, 1\}^\lambda$ of
    a PRF function $ \function{F_K}{\{0, 1\}^*}{\bbZ_p} $ and a random value
    $\alpha \gets_\$ \bbZ_p$. Output $\ek = p$ and $\sk = (K, \alpha)$.
    The message space $\calM$ is defined as $\bbZ_p$.
  \item[$\Auth_\sk(\tau, m) \to \sigma$.] To authenticate a message $m
    \in \bbZ_p$ with label $\tau \in \{0, 1\}^\lambda$, compute $r_\tau
    = F_K(\tau)$, set $y_0 = m$, $y_1 = (r_\tau - m)/\alpha \bmod{p}$ and output
    $\sigma = (y_0, y_1)$. $y_0, y_1$ are the coefficients of a degree-1
    polynomial $y(x)$, where $y(0)$ evaluates to $m$ and $y(\alpha)$ evaluates to
    $r_\tau$.
    
    In this construction, tags $\sigma$ will be interpreted as polynomials
    $y \in \bbZ_p[x]$ of degree $d \geq 1$ in some (unknown) variable
    $x$, i.e., $y(x) = \sum_i{y_i x^i}$ (the value of $d$ is increased by
    successive calls to \Eval).
  \item[$\Eval_\ek(f, \vec{\sigma}) \to \sigma'$.] Receives an
    arithmetic circuit $\function{f}{\bbZ^n_p}{\bbZ_p}$ and a vector of tags
    $\vec{\sigma} = (\sigmaSpace{1}{n})$. Basically, \Eval consists in
    evaluating the circuit $f$ on the tags $\vec{\sigma}$ instead of evaluating
    it on messages. But since the values of $\sigma_i$'s are not valid messages
    in $\bbZ_p$, but rather polynomials in $\bbZ_p[x]$, it is necessary to
    specify how the evaluation should be done.

    \Eval proceeds gate-by-gate, and at each gate $g$, given two tags
    $\sigma_1, \sigma_2$ (or a tag $\sigma_1$ and a constant $c \in \bbZ_p$),
    it runs the sub-routine $\sigma' \gets \GateEval_\ek(g, \sigma_1,
    \sigma_2)$ that returns a new tag $\sigma'$, which is then passed as input
    to the next gate in the circuit.  When the last gate of the circuit is
    reached, \Eval outputs the tag vector $\sigma'$ obtained by running
    \GateEval on the last gate.  \GateEval is described as follows:
    \begin{description}
      \item[$\GateEval_\ek(g, \sigma_1, \sigma_2) \to \sigma'$:] Let
        $\sigma_i = \vec{y}^{(i)} = (\setSpace{y^{(i)}}{0}{d_i})$ for $i = 1,2$
        and $d_i \geq 1$.

        If $g = +$, then:
        \begin{itemize}
          \item let $d = \max(d_1, d_2)$. It is assumed that $d_1 \geq d_2$, so
            $d = d_1$.
          \item Compute the coefficients $(\setSpace{y}{0}{d})$ of the
            polynomial $y(x) = y^{(1)}(x) + y^{(2)}(x)$. This can be done
            efficiently by adding the two vectors of coefficients such that
            $\vec{y} = \vec{y}^{(1)} + \vec{y}^{(2)}$ ($\vec{y}^{(2)}$ is
            eventually padded with zeros in positions $d_1 \dotsc d_2$).
        \end{itemize}
        If $g = \times$, then:
        \begin{itemize}
          \item let $d = d_1 + d_2$.
          \item Compute the coefficients $(\setSpace{y}{0}{d})$ of the
            polynomial $y(x) = y^{(1)}(x) * y^{(2)}(x)$ using the convolution
            operator $*$, i.e., $\forall j = \interval{0}{d} : y_j
            = \sum_{i=0}^{j}{y_i^{(1)} \cdot y_{j-i}^{(2)}}$.
        \end{itemize}
        If $g = \times$ and one of the inputs\footnote{Usually, $\sigma_2$ is the
        constant.} is a constant $c \in \bbZ_p$, then:
        \begin{itemize}
          \item let $d = d_1$.
          \item Compute the coefficients $(\setSpace{y}{0}{d})$ of the
            polynomials $y(x) = c \cdot y^{(1)}(x)$.
        \end{itemize}
        Finally, \GateEval returns $\sigma' = (\setSpace{y}{0}{d})$. The size
        of a tag grows only after the evaluation of a $\times$ gate (where both
        inputs are not constants). After the evaluation of a circuit $f$, it
        holds that $|\sigma'| = \deg(f) + 1$.
    \end{description}
    
  \item[$\Vrfy_\sk(m, \calP, \sigma) \to \{\mathsf{accept, reject}\}$.]
    Let $\calP = f(\tauSpace{1}{n})$ be a labeled program , $m \in \bbZ_p$ and
    $\sigma = (\setSpace{y}{0}{d})$ be a tag for some $d \geq 1$.  Verification
    is done as follows:
    \begin{itemize}
      \item If $y_0 \neq m$, then output \textsf{reject}. Otherwise, continue.
      \item For every input wire of $f$ with label $\tau$ compute $r_\tau
        = F_K(\tau)$. Then compute $\rho = f(\setSpace{r}{\tau_1}{\tau_n})$,
        and use $\alpha$ to check if
        \begin{equation}\label{eq:cf-1-vrfy}
          \rho \equals \sum_{i=0}^{d}{y_i \alpha^i}
        \end{equation}
        If this is true, then output \textsf{accept}. Otherwise, output
        \textsf{reject}\footnote{If using an identity program $\calI_\tau$, the
        \Vrfy just checks that $r_\tau = y_0 + y_1 \cdot \alpha$ and $y_0
        = m$.}.
    \end{itemize}
\end{description}

\paragraph*{Efficiency} The generation of a tag with \Auth is extremely
efficient as it only requires one PRF evaluation (e.g., one AES evaluation).

\Eval's complexity mainly depends on the cost of evaluating the circuit $f$,
and the additional overhead due the \GateEval sub-routine and to that the tag's
size grows with the degree of the circuit. With a circuit of degree $d$, in the
worst case, this overhead is going to be $\calO(d)$ for addition gates, and
$\calO(d \log{d})$ for multiplication gates\footnote{Algorithms based on
  \nom{FFT}{Fast Fourier Transform} can be used to efficiently compute the
  convolution. See \refchapter{chap:algebraic} for details.}

The cost of \Vrfy is basically the cost of computing $\rho$ plus the cost of
computing $\sum_{i=0}^{d}{y_i \alpha^i}$, or $\calO(|f| + d)$.

\paragraph*{Correctness} Essentially, correctness follows from the special
property of the tags generated by \Auth, i.e., that $y(0) = m$ and $y(\alpha)
= r_\tau$. In particular, this property is preserved when evaluating the
circuit $f$ over tags $\sigma_1, \dotsc, \sigma_n$.

\paragraph*{Security} The security of this scheme is based on the following
theorem.
\begin{theorem}
  If $F$ is a PRF, then the previously defined homomorphic MAC scheme is secure.
  \label{first-scheme-sec}
\end{theorem}

\subsubsection*{Compact Homomorphic MAC}
In \citescheme{catalano:fiore:2013}1, the tags' size grows linearly with the
degree of the evaluated circuit. This may be acceptable in some cases, e.g.,
circuits evaluating constant-degree polynomials, but it may become impractical
in other situations, e.g., when the degree is greater that the input size of
the circuit.

To overcome this problem, a second scheme (\citescheme{catalano:fiore:2013}2)
was proposed by \citeauthor{catalano:fiore:2013} that is almost as efficient as
the previous one. An interesting property of this second scheme is that the
tags are now of constant size. But to achieve this a few restrictions arise.
First, there is a fixed a-priori bound $D$ on the degree of allowed circuits.
Second, the homomorphic evaluation has to be done in a ``single shot'' i.e.,
the authentication tags obtained from \Eval cannot be used again to compose
with other tags.

Because of these restrictions, another interesting property, \emph{local
composition}, is achieved. With this, one can keep a locally non-succinct
version of the tag that allows for arbitrary composition. Later, when it comes
to send an authentication tag to the verifier, one can securely compress such
non-succinct tag in a compact one of constant-size.

Besides a PRF, the security of this scheme relies on a re-writing of the
\emph{$\ell$-Diffie-Hellman Inversion} problem. Basically, the definition of
$\ell$-Diffie-Hellman Inversion is that one cannot compute $g^{x^{-1}}$ given
$g, g^\alpha, \dotsc, g^{\alpha^D}$.  The re-write for this scheme states that
one cannot compute $g$ given values $g^\alpha, \dotsc, g^{\alpha^D}$.

The description of the scheme (which we refer to as
\citescheme{catalano:fiore:2013}2) with security parameter $\lambda$ is as
follows:
\begin{description}
  \item[$\KeyGen(1^\lambda, D) \to (\ek, \sk)$: ] Let $D
    = \poly(\lambda)$ be an upper-bound so that the scheme supports circuits of
    degree at most $D$. Generate a group $\bbG$ of order $p$ where $p$ is
    a prime of roughly $\lambda$ bits, and choose a random generator $g
    \gets_\$ \bbG$. Choose a seed $K$ of a PRF $\function{F_K}{\{0,
    1\}^*}{\bbZ_p} $ and a random value $\alpha \gets_\$ \bbZ_p$. For $i
    = \interval{1}{D}$ compute $h_i = g^{\alpha^i}$. Output $\ek
    = (\setSpace{h}{1}{D})$ and $\sk = (K, g, \alpha)$. The message space is
    defined as $\bbZ_p$.
  \item[$\Auth_\sk(\tau, m) \to \sigma$: ] This algorithm is exactly
    the same as the one from the previous construction. Returns a tag $(y_1,
    y_2) \in \bbZ_p^2$.
  \item[$\Eval_\ek(f, \vec{\sigma}) \to \sigma'$: ] Takes as input an
    arithmetic circuit $\function{f}{\bbZ^n_p}{\bbZ_p}$ and a vector of tags
    $\vec{\sigma} = (\sigmaSpace{1}{n})$ such that each $\sigma_i \in \bbZ_p^2$
    (i.e., it is a constant-sized tag for a degree-1 polynomial).  First,
    proceed exactly as in the first scheme to obtain the coefficients
    $(\setSpace{y}{0}{d})$. If $d = 1$ then return $\sigma = (y_0, y_1)$.
    Otherwise, compute $\Lambda = \prod_{i=1}^d{h_i ^{y_i}}$ and return $\sigma
    = \Lambda$.
  \item[$\Vrfy_\sk(m, \calP, \sigma) \to \{\mathsf{accept},
    \mathsf{reject}\}$: ] Let $\calP = (f, \tauSpace{1}{n})$ be a labeled
    program, $m \in \bbZ_p$ and $\sigma$ be a tag of either the form $(y_0,
    y_1) \in \bbZ_p^2$ or $\Lambda \in \bbG$.  First, proceed as in the first
    scheme to compute $\rho$ (i.e., evaluate the circuit with inputs
    $(\setSpace{r}{\tau_1}{\tau_n})$). If $\calP$ computes a polynomial of
    degree 1, then proceed exactly as in the first scheme. Otherwise, use $g$
    to check whether the following holds:
    \begin{equation}\label{eq:cf-2-vrfy}
      g^\rho \equals g^m \cdot \Lambda
    \end{equation}
    If everything is satisfied, then output \textsf{accept}. Otherwise, output
    \textsf{reject}.
\end{description}

\paragraph*{Efficiency} This scheme is practically as efficient as
\citescheme{catalano:fiore:2013}1. Both the tagging and verification algorithms
have exactly the same complexity. In the evaluation, the complexity is still
dominated by the cost of evaluating the circuit and the cost of the sub-routine
\GateEval.

\paragraph*{Correctness} Correctness is achieved much like as in
\citescheme{catalano:fiore:2013}1. Additionally, the
\refequation{eq:cf-2-vrfy} is essentially the same as checking that $\rho
\equals \sum_{i=0}^d y_i \alpha^i$, which is the same as
\refequation{eq:cf-1-vrfy}.

\paragraph*{Security} The security of this scheme is based on the following
theorem.
\begin{theorem}
  If $F$ is a PRF and the $(D-1)$-Diffie Hellman Inversion Assumption holds in
  $\bbG$, then the previously defined homomorphic MAC scheme is secure.
  \label{theo:cf-2-sec}
\end{theorem}

\paragraph*{Local composition} As it was already mentioned,
\citescheme{catalano:fiore:2013}2 introduces local composition. This allows us
to locally keep the large version of the tag (the polynomial $y$ with its $d
+ 1$ coefficients), but the compact version $\Lambda$ is the one that is sent
to the verifier. By keeping this large $y$, it is possible to have composition
as in the previous scheme. This is particularly interesting for applications
where not too many parties are involved, as it allows for succinct tags and
local composition of partial computations at the same time.


%% ---------------------------------------------------------------------------
% BACKES-FIORE-REISCHUCK CONSTRUCTION
%% ---------------------------------------------------------------------------
\subsection{Backes-Fiore-Reischuk}\label{sec:bfr-scheme}
Both \citescheme{catalano:fiore:2013} and \citescheme{gennaro:wichs:2012}
schemes suffer from an inefficient verification algorithm. The time to execute
\Vrfy is proportional to that of executing the program $\calP$, i.e.,
evaluating the circuit $f$. \citeauthor{backes:fiore:reischuk:2013} solved this
problem and additionally, they considered the case of performing computations
over very large sets of inputs. In this case, besides the already mentioned
requirements of security and efficiency, it is desirable to achieve
\emph{input-independent efficiency}, meaning that verifying the correctness of
a computation $f(\mSpace{1}{n})$ requires time \emph{independent} of input size
$n$.  Additionally, two more crucial requirements should be achieved:
\emph{unbounded storage}, meaning that the size of the outsourced data should
not be fixed a-priori; and \emph{function independence}, meaning that a client
should be able to outsource its data without having to know in advance what
functions she will be delegating later.

Input-independent efficiency is achieved in the amortized model: after a first
computation with cost $|f|$, the client can verify every subsequent evaluation
of $f$ in constant time. By fulfilling unbounded storage and function
independence, outsourcing data and function delegation are completely
decoupled: a client can continuously outsource data, and the delegated
functions do not have to fixed a-priori.

Their construction is also limited to support a restricted set of functions,
namely arithmetic circuits of degree up to 2. However, even with this
restricted set it is possible to compute a wide range of interesting
statistical functions, like counting, summation, (weighted) average, arithmetic
mean, standard deviation, variance, co-variance, weighted variance with
constant weights, quadratic mean (RMS), mean squared error (MSE), the Pearson
product-moment correlation coefficient, the coefficient of determination
($R^2$), and the least squares fit of a data set $\{(x_i, v_i)\}^n_{i = 1}$.

In the verification algorithm from both \citescheme{catalano:fiore:2013}
constructions it is always necessary to compute $\rho
= f(\setSpace{r}{\tau_1}{\tau_n})$. One way to accelerate this process would be
to, after the first computation of $\rho$, to re-use it in order to verify
other computations using $f$. However, this involves the re-use of labels,
which is clearly forbidden by the security definition of
\citescheme{catalano:fiore:2013}. The security of their scheme completely
breaks down in the presence of label re-use.

The proposed solution by \citeauthor{backes:fiore:reischuk:2013} to solve this
critical problem involves two new ideas. The first one allows the partial, but
safe, re-use of labels. This is accomplished with the introduction of
multi-labels as previously defined. The other one is a new construction of
a PRF that allows the pre-computation of a piece of label-independent
information $\omega_f$ that can be re-used to efficiently compute $\rho$.
%For the sake of space, the necessary theoretical definitions of such a PRF can
%be found in \refappendix{app:bfr-defs}.

%\chapter{Definitions of \citescheme{backes:fiore:reischuk:2013}}\label{app:bfr-defs}

%In this appendix we provide some theoretical definitions necessary to build the
%\textcite{backes:fiore:reischuk:2013} scheme presented in
%\refsec{sec:bfr-scheme}. We define how homomorphic evaluation is performed
%and what a PRF with Amortized Closed-Form Efficiency is.

Before presenting the actual construction, we introduce some definitions and
utilities necessary to build a PRF with efficient (amortized) verification.
%
%\paragraph*{Homomorphic Evaluation of Arithmetic Circuits}
We start by presenting some basic
%Here we follow~\cite{backes:fiore:reischuk:2013} to present some basic
definitions regarding the homomorphic evaluation of arithmetic circuits over
values defined in some appropriate set $\calJ \neq \calM$. Usually,
$\function{f}{\bbZ_p^n}{\bbZ_p}$, but sometimes the message space may be
defined in $\calJ$, and so it is necessary to obtain an equivalent algorithm
that evaluates messages from $\calJ$ over $f$.

\paragraph*{Homomorphic Evaluation over Polynomials} Consider that
$\calJ_{\poly} = \bbZ_p[x_1, \dotsc, x_m]$ is the ring of polynomials in
variables $x_1, \dotsc, x_m$ over $\bbZ_p$. For every tuple $\vec{a} = (a_1,
\dotsc, a_m) \in \bbZ_p^m$, let
$\function{\phi_{\vec{a}}}{\calJ_{\poly}}{\bbZ_p}$ be the function defined as
$\phi_{\vec{a}}(y) = y(a_1, \dotsc, a_m)$ for any $y \in \calJ_{\poly}$.
$\phi_{\vec{a}}$ is a homomorphism from $\calJ_{\poly}$ to $\bbZ_p$.
%
Given an arithmetic circuit $\function{f}{\bbZ_p^n}{\bbZ_p}$, there exists
another structurally equivalent circuit
$\function{\hat{f}}{\calJ_{\poly}^n}{\calJ_{\poly}}$ such that $\forall y_1,
\dotsc, y_n \in \calJ_{\poly} \colon \phi_{\vec{a}}(\hat{f}(y_1, \dotsc, y_n))
= f(\phi_{\vec{a}}(y_1), \dotsc, \phi_{\vec{a}}(y_n))$. The only difference is
that in every gate the operation in $\bbZ_p$ is replaced by the corresponding
operation in $\bbZ_p[x_1, \dotsc, x_m]$.

More precisely, for every positive integer $m \in \bbN$ and a given $f$, the
computation of $\hat{f}$ on $(y_1, \dotsc, y_n) \in \calJ^n_{\poly}$ can be
defined as $\PolyEval(m, f, y_1, \dotsc, y_n) \to \bbZ_p$. For every gate
$f_g$, on input two polynomials $y_1, y_2 \in \calJ_{\poly}$, proceeds as
follows: if $f_g = +$, it outputs $y = y_1 + y_2$ (adds all coefficients
component-wise); if $f_g = \times$, it outputs $y = y_1 \cdot y_2$ (uses the
convolution operator on the coefficients). Notice that every $\times$ gate
increases the degree $d$ of $y$, as well as its number of coefficients. If
$y_1, y_2$ have degree $d_1, d_2$ respectively, the degree of $y = y_1 \cdot
y_2$ is $d_1 + d_2$.

\paragraph*{Bilinear Groups} Let $\calG(1^\lambda)$ be an algorithm that on
input the security parameter $1^\lambda$, outputs the description of a bilinear
group $\bgpp = (p, \bbG, \bbG_T, e, g)$ where $\bbG$ and $\bbG_T$ are groups of
the same order $p > 2^\lambda$, $g \in \bbG$ is a generator and
$\function{e}{\bbG \times \bbG}{\bbG_T}$ is an efficiently computable bilinear
map, or pairing function. $\calG$ is a \emph{bilinear group generator}.

\paragraph*{Homomorphic Evaluation over Bilinear Groups} Let $\bgpp = (p, \bbG,
\bbG_T, e, g)$ be the description of a bilinear group as previously defined.
With a fixed generator $g \in \bbG$, then $\bbG \cong (\bbZ_p, +)$ (i.e.,
$\bbG$ and the additive group $(\bbZ_p, +)$ are isomorphic) by considering the
isomorphism $\phi_g(x) = g^x$ for every $x \in \bbZ_p$. Similarly, by the
property of the pairing function $e$, $\bbG_T \cong (\bbZ_p, +)$ by considering
the isomorphism $\phi_{g_T}(x) = e(g,g)^x$. Since both $\phi_g$ and
$\phi_{g_T}$ are isomorphisms, there also exist the corresponding inverses
$\function{\phi_g^{-1}}{\bbG}{\bbZ_p}$ and
$\function{\phi_{g_T}^{-1}}{\bbG_T}{\bbZ_p}$, even though these are not known
to be efficiently computable.

For every arithmetic circuit $\function{f}{\bbZ_p^n}{\bbZ_p}$ of degree at most
2, the $\GroupEval(f, X_1, \dotsc, X_n)$ algorithm homomorphically evaluates
$f$ with inputs in $\bbG$ and outputs in $\bbG_T$.

Basically, given a circuit $f$ of degree at most 2, and given an $n$-tuple of
values $(X_1, \dotsc, X_n) \in \bbG^n$, \GroupEval proceeds gate-by-gate as
follows: if $f_g = +$, it uses the group operation in $\bbG$ or in $\bbG_T$; if
$f_g = \times$, it uses the pairing function, thus ``lifting'' the result to
the group $\bbG_T$. By using only circuits of degree at most 2, the
multiplication is well defined.

\GroupEval achieves the desired homomorphic property:
\begin{theorem}\label{theo:group-eval-homo-prop}
  Let $\bgpp = (p, \bbG, \bbG_T, e, g)$ be the description of bilinear groups.
  Then, the algorithm $\GroupEval$ satisfies: $\forall(X_1, \dotsc, X_n) \in
  \bbG^n \colon \GroupEval(f, X_1, \dotsc, X_n) = e(g,g)^{f(x_1, \dotsc, x_n)}$
  for the unique values $\{x_i\}^n_{i=1} \in \bbZ_p$ such that $X_i = g^{x_i}$.
\end{theorem}
The proof of \reftheorem{theo:group-eval-homo-prop} can be found
in~\cite{backes:fiore:reischuk:2013}.

\paragraph*{PRFs with Amortized Closed-Form Efficiency}
A \emph{closed-form efficient} PRF is just like a standard PRF, plus an
additional efficiency property. Consider a computation $\Comp(R_1, \dotsc, R_n,
\vec{z})$ which takes random inputs $R$ and arbitrary inputs $\vec{z}$, and
runs in time $t(n, |\vec{z}|)$. By considering the case where each $R_i
= \mathsf{F}_K(\mathsf{L}_i)$, then the PRF \textsf{F} is said to satisfy
closed-form efficiency for $(\Comp, \vec{\mathsf{L}})$ if, with the knowledge
of the seed $K$, one can compute $\Comp(\mathsf{F}_K(\mathsf{L}_1), \dotsc,
\mathsf{F}_K(\mathsf{L}_n), \vec{z})$ in time strictly less than $t$. The main
idea is that in the pseudo-random case all $R_i$ values have a shorter
closed-form representation (as a function of $K$), and this might also allow
for the existence of a shorter closed-form representation of the computation
\Comp.

From the above considerations it possible to define a new property for PRFs:
\emph{amortized close-form efficiency}. The idea is to address computations
where each $R_i$ is generated as $\mathsf{F}_K(\Delta, \tau_i)$. All the inputs
of \textsf{F} share the same data set label $\Delta$. Then, \textsf{F}
satisfies amortized closed-form efficiency if it is possible to compute $\ell$
computations $\{\Comp(\mathsf{F}_K(\Delta_j, \tau_1), \dotsc,
\mathsf{F}_K(\Delta_j, \tau_n), \vec{z})\}^\ell_{j = 1}$ in time strictly less
than $\ell \cdot t$.

A PRF is defined as follows:
\begin{description}
  \item[$\KeyGen(1^\lambda) \to (\pp, K)$:] Given a security parameter,
    output a secret key $K$ and some public parameters $\pp$ that specify
    domain $\calX$ and range $\calR$ of the function.
  \item[$\F_K(x) \to R$:] Receives $x \in \calX$ and uses the secret
    key $K$ to compute a value $R \in \calR$.
\end{description}
A PRF must satisfy the pseudo-randomness property. More precisely, (\KeyGen, \F)
is \emph{secure} if for every PPT adversary $\calA$ it holds:
\begin{equation}\label{eq:amor-closed-form-prf-sec}
  \abs*{\Pr[\calA^{\F_K(\cdot)}(1^\lambda, \pp) = 1]
  - \Pr[\calA^{\phi(\cdot)}(1^\lambda, \pp) = 1]} \leq \negl(\lambda)
\end{equation}
$(K, \pp) \gets_\$ \KeyGen(1^\lambda)$, and
$\function{\phi}{\calX}{\calR}$ is a random function.

\begin{definition}\label{def:amort-closed-form-eff-prf}
  Consider a computation $\Comp(R_1, \dotsc, R_n, z_1, \dotsc, z_m)$ with
  random $n$ input values and $m$ arbitrary input values. \Comp takes time
  $t(n,m)$. Let $\vec{\mathsf{L}} = (\mathsf{L}_1, \dotsc, \mathsf{L}_n)$ be
  arbitrary values in the domain $\calX$ of \textsf{F} such that $\mathsf{L}_i
  = (\Delta, \tau_i)$. A PRF (\KeyGen, \F) satisfies \emph{amortized
  closed-form efficiency} for $(\Comp, \vec{\mathsf{L}})$ if there exist
  algorithms $\CFEval^\off_{\Comp, \vec{\tau}}$ and $\CFEval^\on_{\Comp,
  \Delta}$ such that:
  \begin{enumerate}
    \item Given $\omega \gets \CFEval^\off_{\Comp, \vec{\tau}}(K,
      \vec{z})$, then
      \begin{equation}\label{eq:prf-acf-eff-1}
        \CFEval^\on_{\Comp, \Delta}(K, \omega) = \Comp(\F_K(\Delta, \tau_1),
        \dotsc, \F_K(\Delta, \tau_n), z_1, \dotsc, z_m)
      \end{equation}
    \item The running time of $\CFEval^\on_{\Comp, \Delta}(K, \omega)$ is
      $o(t)$.
  \end{enumerate}
\end{definition}
The $\omega$ obtained from the offline computation $\CFEval^\off_{\Comp, \tau}$
does not depend on data set label $\Delta$, which means that it can be re-used
in further online computations $\CFEval^\on_{\Comp, \Delta}(K, \omega)$ to
compute $\Comp(\F_K(\Delta, \tau_1), \dotsc, \F_K(\Delta, \tau_n), \vec{z})$
for many different $\Delta$'s.
%
Because of this, the efficiency property puts a restriction only on
$\CFEval^\on_{\Comp,\Delta}$ in order to capture the idea of achieving
efficiency in an amortized sense when considering many evaluations over of
$\Comp(\F_K(\Delta, \tau_1), \dotsc, \F_K(\Delta, \tau_n), \dotsc, \vec{z})$,
with different data set label $\Delta$ in each evaluation. Essentially, one can
pre-compute $\omega$ once, and then use it to run the online phase as many
times as needed, almost for free.

The structure of \Comp may impose some restrictions on the range $\calR$ of the
PRF, and due to the pseudo-randomness property, the output distribution of
$\CFEval^\on_{\Comp, \Delta}(K, \CFEval^\off_{\Comp, \vec{\tau}}(K, \vec{z}))$
is computationally indistinguishable from the output distribution of
$\Comp(R_1, \dotsc, R_n, \vec{z})$.


\paragraph*{Amortized Closed-Form Efficient PRF} Before introducing the
description of \citescheme{backes:fiore:reischuk:2013} scheme, it is necessary
to define this new and more efficient PRF, namely a PRF with amortized
closed-form efficiency for \GroupEval. This PRF uses two generic PRFs which map
binary strings to integers in $\bbZ_p$, and a weak PRF whose security relies on
the Decision Linear assumption introduced by
\textcite{boneh:boyen:shacham:2004}, and is presented below:
\begin{definition}
  Let $\calG$ bilinear group generator, and $\bgpp = (p, \bbG, \bbG_T, e, g)
  \gets_\$ \calG(1^\lambda)$. Let $\tuple{g_0, g_1, g_2} \gets_\$ \bbG$ and
  $\tuple{r_0, r_1, r_2} \gets_\$ \bbZ_p$ be chosen uniformly at random. The
  advantage of a PPT adversary $\calA$ in solving the \emph{Decision Linear}
  problem is
  \begin{align}
    \mathbf{Adv}^{DLin}_\calA(\lambda) = |&\Pr[\calA(\bgpp, g_0, g_1, g_2,
    g_1^{r_1}, g_2^{r_2}, g_0^{r_1 + r2}) = 1] - \nonumber \\
    & \Pr[\calA(\bgpp, g_0, g_1,
    g_2, g_1^{r_1}, g_2^{r_2}, g_0^{r_0}] = 1|
  \end{align}
  The Decision Linear assumption holds for $\calG$ if for every PPT adversary
  $\calA$ if $\mathbf{Adv}^{DLin}_\calA(\lambda) \leq \negl(\lambda)$.
\end{definition}

\paragraph*{Syntax}The syntax of an amortized closed-form efficient PRF is as
follows:
\begin{description}
  \item[$\KeyGen(1^\lambda) \to (\pp, K)$: ] Let $\bgpp = (p, \bbG,
    \bbG_T, e, g)$ be the description of bilinear groups $\bbG$ and $\bbG_T$
    having the same prime order $p > 2^\lambda$ and such that $g \in \bbG$ is
    a generator, and $\function{e}{\bbG \times \bbG}{\bbG_T}$ is an efficiently
    computable bilinear map. Choose two seeds $K_1, K_2$ for a family of PRFs
    $\function{\F'_{K_i}}{\{0,1\}^*}{\bbZ_p^2}$ for $i = 1,2$. Output
    $K = (\bgpp, K_1, K_2)$ and $\pp = \bgpp$.
  \item[$\F_K(x) \to R$:] Let $x = (\Delta, \tau) \in \calX$. To
    compute $R \in \bbG$, generate $(u,v) \gets \F'_{K_1}(\tau)$ and
    $(a,b) = \F'_{K_2}(\Delta)$, and output $R = g^{ua + vb}$.
\end{description}
The previous PRF is pseudo-random. The proof of the following theorem can be
found in~\cite{backes:fiore:reischuk:2013}.

\begin{theorem}\label{theo:amort-closed-form-prf}
  If $\F'$ is a PRF and the Decision Linear assumptions holds for $\calG$, then
  the function $(\tuple{\KeyGen, \F})$ presented above is a PRF.
\end{theorem}

\paragraph*{Amortized Closed-Form Efficiency for \GroupEval} The previously
defined PRF satisfies amortized closed-form efficiency for $(\GroupEval,
\vec{\mathsf{L}})$. Recall that $\function{\GroupEval}{f \times
\bbG^n}{\bbG_T}$, with arithmetic circuit $\function{f}{\bbZ^n_p}{\bbZ_p}$ and
$R_1, \dotsc, R_n \in \bbG$ random values. $\vec{\mathsf{L}}$ is a vector
$(\mathsf{L}_1, \dotsc, \mathsf{L}_n)$ with each $\mathsf{L}_i = (\Delta,
\tau_i) \in \calX$.
\begin{description}
  \item[$\CFEval^\off_{\GroupEval, \vec{\tau}}(K, f) \to \omega_f$:]
    Let $K = (\bgpp, K_1, K_2)$ be a secret key generated by
    $\KeyGen(1^\lambda)$ of a closed-form efficient PRF. Compute $\{(u_i, v_i)
      \gets \F'_{K_1}(\tau_i)\}_{i=1}^{n}$ and set $\vec{\rho_i} = (0,
      u_i, v_i)$. Basically, $\vec{\rho_i}$ are the coefficients of a degree-1
      polynomial $\rho_i(x_1, x_2)$ in two unknown variables $x_1, x_2$. Then,
      run $\PolyEval(2, f, \rho_1, \dotsc, \rho_n)$ to compute the coefficients
      $\vec{\rho}$ of a polynomial $\rho(x_1, x_2)$ such that $\forall x_1, x_2
      \in \bbZ_p \colon \rho(x_1, x_2) = f(\rho_1(x_1, x_2), \dotsc,
      \rho_n(x_1, x_2))$. Output $\omega_f = \vec{\rho}$.
  \item[$\CFEval^\on_{\GroupEval, \Delta}(K, \omega_f) \to W$:] Let $K
    = (\bgpp, K_1, K_2)$ be a secret key and $\omega_f = \vec{\rho}$ the result
    of the previous algorithm. Generate $(a,b) \gets \F'_{K_2}(\Delta)$,
    and then use the coefficients $\vec{\rho}$ to compute $w = \rho(a,b)$.
    Output $W = e(g, g)^w$.
\end{description}

\begin{theorem}
  Let $\vec{\mathsf{L}} = (\mathsf{L}_1, \dotsc, \mathsf{L}_n)$ be such that
  $\mathsf{L}_i = (\Delta, \tau_i) \in \calX$ and let $t$ be the running time
  of $\GroupEval$. Then the PRF $(\KeyGen, \F)$, extended with the algorithms
  $\CFEval^\off_{\GroupEval, \vec{\tau}}$ and $\CFEval^\on_{\GroupEval,
  \Delta}$ satisfies \emph{amortized closed-form efficiency for $(\GroupEval,
    \vec{\mathsf{L}})$} according to \refdef{def:amort-closed-form-eff-prf}, with running
    times of $\calO(t)$ and $\calO(1)$ for $\CFEval^\off_{\GroupEval,
      \vec{\tau}}$ and $\CFEval^\on_{\GroupEval, \Delta}$, respectively.
  \label{theo:amort-closed-form-group-eval}
\end{theorem}

The proof can be found in \cite{backes:fiore:reischuk:2013}.

\paragraph*{Homomorphic MAC with Efficient Verification} Having defined the PRF
that allows to more efficiently compute the $\rho$ from the verification
algorithm, we can now present the construction of an homomorphic MAC scheme
with an efficient verification algorithm.

\citescheme{backes:fiore:reischuk:2013} scheme works for circuits whose
additive gates do not get inputs labeled by constants. And just like in the
\citescheme{catalano:fiore:2013}, without loss of generality, one can use an
equivalent circuit where there is a special variable/label for the value of 1,
and the MAC of 1 can be published.  The description of the $\EVHMAC$
construction is as follows:
\begin{description}
  \item[$\KeyGen(1^\lambda) \to (\ek, \sk)$:] Run $\bgpp \gets_\$
    \calG(1^\lambda)$ to generate the description of bilinear groups with
    $\bgpp = (p, \bbG, \bbG_T, e, g)$. The message space $\calM$ is $\bbZ_p$.
    Choose a random value $\alpha \gets_\$ \bbZ_p$, and run $(K, \pp) \gets_\$
    \KeyGen_(1^\lambda)$ to obtain seed $K$ of a PRF function
    $\function{\mathsf{F}_K}{\{0,1\}^* \times \{0,1\}^*}{\bbG}$.  Output $\sk
    = (\bgpp, \pp, K, \alpha)$ and $\ek = (\bgpp, \pp)$.
  \item[$\Auth_\sk(\mathsf{L}, m) \to \sigma$:] Receives a multi-label
    $(\Delta, \tau) \in (\{0,1\}^\lambda)^2$ and a message $m \in \calM$.
    Compute $R \gets \mathsf{F}_K(\Delta, \tau)$ and set $y_0 = m$ and
    $Y_1 = (R \cdot g^{-m})^{1/\alpha}$. Output the tag $(y_0, Y_1) \in \bbZ_p
    \times \bbG$.

    If we consider the $y_1 \in \bbZ_p$ to be the unique value such that
    $Y_1 = g^{y_1}$, then $(y_0, y_1)$ are simply the coefficients of
    a degree-1 polynomial $y(x)$ such that $y(0) = m$ and $y(\alpha)
    = \phi_g^{-1}(R)$.
  \item[$\Eval_\ek(f, \vec{\sigma} \to \sigma')$:] Receives a vector of
    tags $\sigmaSpace{1}{n}$ and an arithmetic circuit
    $\function{f}{\bbZ_p^n}{\bbZ_p}$. Just as in
    \citescheme{catalano:fiore:2013} scheme, the values of $\sigma_i$'s are
    not valid messages in $\bbZ_p$, and so it is necessary to specify the
    sub-routine $\GateEval$.
    At a given gate $f_g$, given two tags $\sigma_1, \sigma_2$, $\GateEval$
    proceeds as follows:
    \begin{description}
      \item[$\GateEval_\ek(f_g, \sigma_1, \sigma_2) \to \sigma'$:] Let
        $\sigma_i = (y_0^{(i)}, Y_1^{(i)}, \hat{Y}_2^{(i)}) \in \bbZ_p \times
        \bbG \times \bbG_T$ for $i = 1,2$. Assume that $\hat{Y}_2^{(i)} = 1$
        whenever it is not defined.

        If $f_g = +$, then compute $(y_0, Y_1, \hat{Y}_2)$ as:
        \begin{itemize}
          \item $y_0 = y_0^{(1)} + y_0^{(2)}$
          \item $Y_1 = Y_1^{(1)} \cdot Y_1^{(2)}$
          \item $\hat{Y}_2 = \hat{Y}_2^{(1)} \cdot \hat{Y}_2^{(2)}$
        \end{itemize}
        If $f_g = \times$, then compute $(y_0, Y_1, \hat{Y}_2)$ as:
        \begin{itemize}
          \item $y_0 = y_0^{(1)} \cdot y_0^{(2)}$
          \item $Y_1 = (Y_1^{(1)})^{y_0^{(2)}} \cdot
            (Y_1^{(2)})^{y_0^{(1)}}$
          \item $\hat{Y}_2 = e(Y_1^{(1)}, Y_1^{(2)})$

        Since this construction assumes that $deg(f) \leq 2$, it can be assumed
        that $\sigma_i = (\tuple{y_0^{(i)}, Y_1^{(i)}}) \in \bbZ_p \times \bbG$ for
        $i = 1,2$.
        \end{itemize}
        If $f_g = \times$ and one of the two inputs is a constant $c \in
        \bbZ_p$ (assume that $\sigma_2$ is the constant), then compute $(y_0,
        Y_1, \hat{Y}_2)$ as:
        \begin{itemize}
          \item $y_0 = c \cdot y_0^{(1)}$
          \item $Y_1 = (Y_1^{(1)})^c$
          \item $\hat{Y}_2 = (\hat{Y}_2^{(1)})^c$
        \end{itemize}
    \end{description}
    Return $\sigma' = (y_0, Y_1, \hat{Y}_2)$.
  \item[$\Vrfy_\sk(m, \calP_\Delta, \sigma) \to \{\mathsf{accept},
    \mathsf{reject}\}$:] Receives a message $m \in \calM$, a multi-labeled
    program $\calP_\Delta = (f, \tauSpace{1}{n}), \Delta)$ and a tag
    $\sigma = (y_0, Y_1, \hat{Y}_2)$. For $i = 1$ to $n$, compute $R_i
    \gets \mathsf{F}_K(\Delta, \tau_i)$. Then run $W \gets
    \GroupEval(f, \setSpace{R}{1}{n}) \in \bbG_T$, and check that both:
    \begin{align*}
      m & \equals y_0\\
      W & \equals e(g,g)^{y_0} \cdot e(Y_1, g)^\alpha \cdot
      (\hat{Y}_2)^{\alpha^2}
    \end{align*}
    If both checks are satisfied, output \textsf{accept}. Output
    \textsf{reject} otherwise.
\end{description}

The previous verification algorithm is still not efficient. The description of
$\EVHMAC$ needs to be complemented with \VrfyPrep and \EffVrfy to achieve
efficient verification.
\begin{description}
  \item[$\VrfyPrep_\sk(\calP) \to \vk_\calP$:] Let $\sk = (\bgpp, \pp, K,
    \alpha)$. Receives a labeled program $\calP = (f, \tauSpace{1}{n})$.
    A concise verification key $\vk_\calP = \omega$ is computed, with $\omega
    \gets \CFEval^\off_{\GroupEval, \vec{\tau}}(K, f)$.
  \item[$\EffVrfy_{\sk, \vk_\calP}(\Delta, m, \sigma) \to \{\mathsf{accept},
    \mathsf{reject}\}$:] Let $sk = (\bgpp, \pp, K, \alpha)$ and $\vk_\calP =
    \omega$ and $\sigma = (\tuple{y_0, Y_1, \hat{Y}_2})$. First, run the online
    closed-form efficient algorithm of \F for \GroupEval to compute $W \gets
    \GroupEval^\on_{\GroupEval, \Delta}(K, \omega)$. Then, it runs the same
    checks as in \Vrfy, and returns \textsf{accept} if both checks are
    satisfied, \textsf{false} otherwise.
\end{description}

\paragraph*{Correctness} $\EVHMAC$ satisfies both authentication and evaluating
correctness. The respective proofs can be found
in~\cite{backes:fiore:reischuk:2013}.

\paragraph*{Efficient Verification} $\EVHMAC$ satisfies efficient verification.
\begin{theorem}
  If $\F$ has amortized closed-form efficiency for $(\GroupEval,
  \vec{\mathsf{L}})$, then $\EVHMAC$ satisfies efficient verification.
  \label{theo:evh-mac-eff-vrfy}
\end{theorem}
The verification preparation $\VrfyPrep$ runs in time equal to
$\CFEval^\off_{\GroupEval,\vec{\tau}}$, and the efficient verification
$\EffVrfy$ runs in time equal to $\CFEval^\on_{\GroupEval, \Delta}$. From
\reftheorem{theo:amort-closed-form-group-eval}, \VrfyPrep and \EffVrfy run in time
$\calO(\abs{f})$ and $\calO(1)$, respectively. The full proof can be found
in~\cite{backes:fiore:reischuk:2013}.

\paragraph*{Security} The security of $\EVHMAC$ is based on the following
theorem.
\begin{theorem}
  Let $\lambda$ be the security parameter, $\F$ be a PRF with security
  $\epsilon_\F$, and $\calG$ be a bilinear group generator. Then, any PPT
  adversary $\calA$ making $Q$ verification queries has at most probability
  \begin{equation}\label{eq:evh-mac-sec}
    \Pr[\mathsf{HomUF-CMA}_{\calA, \HMACML} = 1] \leq 2 \cdot \epsilon_\F
    + \frac{8Q}{p-2(Q-1)}
  \end{equation}
  of breaking the security of $\EVHMAC$.
  \label{theo:evh-mac-sec}
\end{theorem}


