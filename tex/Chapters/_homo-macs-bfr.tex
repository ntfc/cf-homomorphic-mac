%% ---------------------------------------------------------------------------
% BACKES-FIORE-REISCHUCK CONSTRUCTION
%% ---------------------------------------------------------------------------
\subsection{Backes-Fiore-Reischuk}\label{sec:bfr-scheme}
Both \citescheme{catalano:fiore:2013} and \citescheme{gennaro:wichs:2012}
schemes suffer from an inefficient verification algorithm. The time to execute
\Vrfy is proportional to that of executing the program $\calP$, i.e.,
evaluating the circuit $f$. \citeauthor{backes:fiore:reischuk:2013} solved this
problem and additionally, they considered the case of performing computations
over very large sets of inputs. In this case, besides the already mentioned
requirements of security and efficiency, it is desirable to achieve
\emph{input-independent efficiency}, meaning that verifying the correctness of
a computation $f(\mSpace{1}{n})$ requires time \emph{independent} of input size
$n$.  Additionally, two more crucial requirements should be achieved:
\emph{unbounded storage}, meaning that the size of the outsourced data should
not be fixed a-priori; and \emph{function independence}, meaning that a client
should be able to outsource its data without having to know in advance what
functions she will be delegating later.

Input-independent efficiency is achieved in the amortized model: after a first
computation with cost $|f|$, the client can verify every subsequent evaluation
of $f$ in constant time. By fulfilling unbounded storage and function
independence, outsourcing data and function delegation are completely
decoupled: a client can continuously outsource data, and the delegated
functions do not have to fixed a-priori.

Their construction is also limited to support a restricted set of functions,
namely arithmetic circuits of degree up to 2. However, even with this
restricted set it is possible to compute a wide range of interesting
statistical functions, like counting, summation, (weighted) average, arithmetic
mean, standard deviation, variance, co-variance, weighted variance with
constant weights, quadratic mean (RMS), mean squared error (MSE), the Pearson
product-moment correlation coefficient, the coefficient of determination
($R^2$), and the least squares fit of a data set $\{(x_i, v_i)\}^n_{i = 1}$.

In the verification algorithm from both \citescheme{catalano:fiore:2013}
constructions it is always necessary to compute $\rho
= f(\setSpace{r}{\tau_1}{\tau_n})$. One way to accelerate this process would be
to, after the first computation of $\rho$, to re-use it in order to verify
other computations using $f$. However, this involves the re-use of labels,
which is clearly forbidden by the security definition of
\citescheme{catalano:fiore:2013}. The security of their scheme completely
breaks down in the presence of label re-use.

The proposed solution by \citeauthor{backes:fiore:reischuk:2013} to solve this
critical problem involves two new ideas. The first one allows the partial, but
safe, re-use of labels. This is accomplished with the introduction of
multi-labels as previously defined. The other one is a new construction of
a PRF that allows the pre-computation of a piece of label-independent
information $\omega_f$ that can be re-used to efficiently compute $\rho$.
%For the sake of space, the necessary theoretical definitions of such a PRF can
%be found in \refappendix{app:bfr-defs}.

%\chapter{Definitions of \citescheme{backes:fiore:reischuk:2013}}\label{app:bfr-defs}

%In this appendix we provide some theoretical definitions necessary to build the
%\textcite{backes:fiore:reischuk:2013} scheme presented in
%\refsec{sec:bfr-scheme}. We define how homomorphic evaluation is performed
%and what a PRF with Amortized Closed-Form Efficiency is.

Before presenting the actual construction, we introduce some definitions and
utilities necessary to build a PRF with efficient (amortized) verification.
%
%\paragraph*{Homomorphic Evaluation of Arithmetic Circuits}
We start by presenting some basic
%Here we follow~\cite{backes:fiore:reischuk:2013} to present some basic
definitions regarding the homomorphic evaluation of arithmetic circuits over
values defined in some appropriate set $\calJ \neq \calM$. Usually,
$\function{f}{\bbZ_p^n}{\bbZ_p}$, but sometimes the message space may be
defined in $\calJ$, and so it is necessary to obtain an equivalent algorithm
that evaluates messages from $\calJ$ over $f$.

\paragraph*{Homomorphic Evaluation over Polynomials} Consider that
$\calJ_{\poly} = \bbZ_p[x_1, \dotsc, x_m]$ is the ring of polynomials in
variables $x_1, \dotsc, x_m$ over $\bbZ_p$. For every tuple $\vec{a} = (a_1,
\dotsc, a_m) \in \bbZ_p^m$, let
$\function{\phi_{\vec{a}}}{\calJ_{\poly}}{\bbZ_p}$ be the function defined as
$\phi_{\vec{a}}(y) = y(a_1, \dotsc, a_m)$ for any $y \in \calJ_{\poly}$.
$\phi_{\vec{a}}$ is a homomorphism from $\calJ_{\poly}$ to $\bbZ_p$.
%
Given an arithmetic circuit $\function{f}{\bbZ_p^n}{\bbZ_p}$, there exists
another structurally equivalent circuit
$\function{\hat{f}}{\calJ_{\poly}^n}{\calJ_{\poly}}$ such that $\forall y_1,
\dotsc, y_n \in \calJ_{\poly} \colon \phi_{\vec{a}}(\hat{f}(y_1, \dotsc, y_n))
= f(\phi_{\vec{a}}(y_1), \dotsc, \phi_{\vec{a}}(y_n))$. The only difference is
that in every gate the operation in $\bbZ_p$ is replaced by the corresponding
operation in $\bbZ_p[x_1, \dotsc, x_m]$.

More precisely, for every positive integer $m \in \bbN$ and a given $f$, the
computation of $\hat{f}$ on $(y_1, \dotsc, y_n) \in \calJ^n_{\poly}$ can be
defined as $\PolyEval(m, f, y_1, \dotsc, y_n) \to \bbZ_p$. For every gate
$f_g$, on input two polynomials $y_1, y_2 \in \calJ_{\poly}$, proceeds as
follows: if $f_g = +$, it outputs $y = y_1 + y_2$ (adds all coefficients
component-wise); if $f_g = \times$, it outputs $y = y_1 \cdot y_2$ (uses the
convolution operator on the coefficients). Notice that every $\times$ gate
increases the degree $d$ of $y$, as well as its number of coefficients. If
$y_1, y_2$ have degree $d_1, d_2$ respectively, the degree of $y = y_1 \cdot
y_2$ is $d_1 + d_2$.

\paragraph*{Bilinear Groups} Let $\calG(1^\lambda)$ be an algorithm that on
input the security parameter $1^\lambda$, outputs the description of a bilinear
group $\bgpp = (p, \bbG, \bbG_T, e, g)$ where $\bbG$ and $\bbG_T$ are groups of
the same order $p > 2^\lambda$, $g \in \bbG$ is a generator and
$\function{e}{\bbG \times \bbG}{\bbG_T}$ is an efficiently computable bilinear
map, or pairing function. $\calG$ is a \emph{bilinear group generator}.

\paragraph*{Homomorphic Evaluation over Bilinear Groups} Let $\bgpp = (p, \bbG,
\bbG_T, e, g)$ be the description of a bilinear group as previously defined.
With a fixed generator $g \in \bbG$, then $\bbG \cong (\bbZ_p, +)$ (i.e.,
$\bbG$ and the additive group $(\bbZ_p, +)$ are isomorphic) by considering the
isomorphism $\phi_g(x) = g^x$ for every $x \in \bbZ_p$. Similarly, by the
property of the pairing function $e$, $\bbG_T \cong (\bbZ_p, +)$ by considering
the isomorphism $\phi_{g_T}(x) = e(g,g)^x$. Since both $\phi_g$ and
$\phi_{g_T}$ are isomorphisms, there also exist the corresponding inverses
$\function{\phi_g^{-1}}{\bbG}{\bbZ_p}$ and
$\function{\phi_{g_T}^{-1}}{\bbG_T}{\bbZ_p}$, even though these are not known
to be efficiently computable.

For every arithmetic circuit $\function{f}{\bbZ_p^n}{\bbZ_p}$ of degree at most
2, the $\GroupEval(f, X_1, \dotsc, X_n)$ algorithm homomorphically evaluates
$f$ with inputs in $\bbG$ and outputs in $\bbG_T$.

Basically, given a circuit $f$ of degree at most 2, and given an $n$-tuple of
values $(X_1, \dotsc, X_n) \in \bbG^n$, \GroupEval proceeds gate-by-gate as
follows: if $f_g = +$, it uses the group operation in $\bbG$ or in $\bbG_T$; if
$f_g = \times$, it uses the pairing function, thus ``lifting'' the result to
the group $\bbG_T$. By using only circuits of degree at most 2, the
multiplication is well defined.

\GroupEval achieves the desired homomorphic property:
\begin{theorem}\label{theo:group-eval-homo-prop}
  Let $\bgpp = (p, \bbG, \bbG_T, e, g)$ be the description of bilinear groups.
  Then, the algorithm $\GroupEval$ satisfies: $\forall(X_1, \dotsc, X_n) \in
  \bbG^n \colon \GroupEval(f, X_1, \dotsc, X_n) = e(g,g)^{f(x_1, \dotsc, x_n)}$
  for the unique values $\{x_i\}^n_{i=1} \in \bbZ_p$ such that $X_i = g^{x_i}$.
\end{theorem}
The proof of \reftheorem{theo:group-eval-homo-prop} can be found
in~\cite{backes:fiore:reischuk:2013}.

\paragraph*{PRFs with Amortized Closed-Form Efficiency}
A \emph{closed-form efficient} PRF is just like a standard PRF, plus an
additional efficiency property. Consider a computation $\Comp(R_1, \dotsc, R_n,
\vec{z})$ which takes random inputs $R$ and arbitrary inputs $\vec{z}$, and
runs in time $t(n, |\vec{z}|)$. By considering the case where each $R_i
= \mathsf{F}_K(\mathsf{L}_i)$, then the PRF \textsf{F} is said to satisfy
closed-form efficiency for $(\Comp, \vec{\mathsf{L}})$ if, with the knowledge
of the seed $K$, one can compute $\Comp(\mathsf{F}_K(\mathsf{L}_1), \dotsc,
\mathsf{F}_K(\mathsf{L}_n), \vec{z})$ in time strictly less than $t$. The main
idea is that in the pseudo-random case all $R_i$ values have a shorter
closed-form representation (as a function of $K$), and this might also allow
for the existence of a shorter closed-form representation of the computation
\Comp.

From the above considerations it possible to define a new property for PRFs:
\emph{amortized close-form efficiency}. The idea is to address computations
where each $R_i$ is generated as $\mathsf{F}_K(\Delta, \tau_i)$. All the inputs
of \textsf{F} share the same data set label $\Delta$. Then, \textsf{F}
satisfies amortized closed-form efficiency if it is possible to compute $\ell$
computations $\{\Comp(\mathsf{F}_K(\Delta_j, \tau_1), \dotsc,
\mathsf{F}_K(\Delta_j, \tau_n), \vec{z})\}^\ell_{j = 1}$ in time strictly less
than $\ell \cdot t$.

A PRF is defined as follows:
\begin{description}
  \item[$\KeyGen(1^\lambda) \to (\pp, K)$:] Given a security parameter,
    output a secret key $K$ and some public parameters $\pp$ that specify
    domain $\calX$ and range $\calR$ of the function.
  \item[$\F_K(x) \to R$:] Receives $x \in \calX$ and uses the secret
    key $K$ to compute a value $R \in \calR$.
\end{description}
A PRF must satisfy the pseudo-randomness property. More precisely, (\KeyGen, \F)
is \emph{secure} if for every PPT adversary $\calA$ it holds:
\begin{equation}\label{eq:amor-closed-form-prf-sec}
  \abs*{\Pr[\calA^{\F_K(\cdot)}(1^\lambda, \pp) = 1]
  - \Pr[\calA^{\phi(\cdot)}(1^\lambda, \pp) = 1]} \leq \negl(\lambda)
\end{equation}
$(K, \pp) \gets_\$ \KeyGen(1^\lambda)$, and
$\function{\phi}{\calX}{\calR}$ is a random function.

\begin{definition}\label{def:amort-closed-form-eff-prf}
  Consider a computation $\Comp(R_1, \dotsc, R_n, z_1, \dotsc, z_m)$ with
  random $n$ input values and $m$ arbitrary input values. \Comp takes time
  $t(n,m)$. Let $\vec{\mathsf{L}} = (\mathsf{L}_1, \dotsc, \mathsf{L}_n)$ be
  arbitrary values in the domain $\calX$ of \textsf{F} such that $\mathsf{L}_i
  = (\Delta, \tau_i)$. A PRF (\KeyGen, \F) satisfies \emph{amortized
  closed-form efficiency} for $(\Comp, \vec{\mathsf{L}})$ if there exist
  algorithms $\CFEval^\off_{\Comp, \vec{\tau}}$ and $\CFEval^\on_{\Comp,
  \Delta}$ such that:
  \begin{enumerate}
    \item Given $\omega \gets \CFEval^\off_{\Comp, \vec{\tau}}(K,
      \vec{z})$, then
      \begin{equation}\label{eq:prf-acf-eff-1}
        \CFEval^\on_{\Comp, \Delta}(K, \omega) = \Comp(\F_K(\Delta, \tau_1),
        \dotsc, \F_K(\Delta, \tau_n), z_1, \dotsc, z_m)
      \end{equation}
    \item The running time of $\CFEval^\on_{\Comp, \Delta}(K, \omega)$ is
      $o(t)$.
  \end{enumerate}
\end{definition}
The $\omega$ obtained from the offline computation $\CFEval^\off_{\Comp, \tau}$
does not depend on data set label $\Delta$, which means that it can be re-used
in further online computations $\CFEval^\on_{\Comp, \Delta}(K, \omega)$ to
compute $\Comp(\F_K(\Delta, \tau_1), \dotsc, \F_K(\Delta, \tau_n), \vec{z})$
for many different $\Delta$'s.
%
Because of this, the efficiency property puts a restriction only on
$\CFEval^\on_{\Comp,\Delta}$ in order to capture the idea of achieving
efficiency in an amortized sense when considering many evaluations over of
$\Comp(\F_K(\Delta, \tau_1), \dotsc, \F_K(\Delta, \tau_n), \dotsc, \vec{z})$,
with different data set label $\Delta$ in each evaluation. Essentially, one can
pre-compute $\omega$ once, and then use it to run the online phase as many
times as needed, almost for free.

The structure of \Comp may impose some restrictions on the range $\calR$ of the
PRF, and due to the pseudo-randomness property, the output distribution of
$\CFEval^\on_{\Comp, \Delta}(K, \CFEval^\off_{\Comp, \vec{\tau}}(K, \vec{z}))$
is computationally indistinguishable from the output distribution of
$\Comp(R_1, \dotsc, R_n, \vec{z})$.


\paragraph*{Amortized Closed-Form Efficient PRF} Before introducing the
description of \citescheme{backes:fiore:reischuk:2013} scheme, it is necessary
to define this new and more efficient PRF, namely a PRF with amortized
closed-form efficiency for \GroupEval. This PRF uses two generic PRFs which map
binary strings to integers in $\bbZ_p$, and a weak PRF whose security relies on
the Decision Linear assumption introduced by
\textcite{boneh:boyen:shacham:2004}, and is presented below:
\begin{definition}
  Let $\calG$ bilinear group generator, and $\bgpp = (p, \bbG, \bbG_T, e, g)
  \gets_\$ \calG(1^\lambda)$. Let $\tuple{g_0, g_1, g_2} \gets_\$ \bbG$ and
  $\tuple{r_0, r_1, r_2} \gets_\$ \bbZ_p$ be chosen uniformly at random. The
  advantage of a PPT adversary $\calA$ in solving the \emph{Decision Linear}
  problem is
  \begin{align}
    \mathbf{Adv}^{DLin}_\calA(\lambda) = |&\Pr[\calA(\bgpp, g_0, g_1, g_2,
    g_1^{r_1}, g_2^{r_2}, g_0^{r_1 + r2}) = 1] - \nonumber \\
    & \Pr[\calA(\bgpp, g_0, g_1,
    g_2, g_1^{r_1}, g_2^{r_2}, g_0^{r_0}] = 1|
  \end{align}
  The Decision Linear assumption holds for $\calG$ if for every PPT adversary
  $\calA$ if $\mathbf{Adv}^{DLin}_\calA(\lambda) \leq \negl(\lambda)$.
\end{definition}

\paragraph*{Syntax}The syntax of an amortized closed-form efficient PRF is as
follows:
\begin{description}
  \item[$\KeyGen(1^\lambda) \to (\pp, K)$: ] Let $\bgpp = (p, \bbG,
    \bbG_T, e, g)$ be the description of bilinear groups $\bbG$ and $\bbG_T$
    having the same prime order $p > 2^\lambda$ and such that $g \in \bbG$ is
    a generator, and $\function{e}{\bbG \times \bbG}{\bbG_T}$ is an efficiently
    computable bilinear map. Choose two seeds $K_1, K_2$ for a family of PRFs
    $\function{\F'_{K_i}}{\{0,1\}^*}{\bbZ_p^2}$ for $i = 1,2$. Output
    $K = (\bgpp, K_1, K_2)$ and $\pp = \bgpp$.
  \item[$\F_K(x) \to R$:] Let $x = (\Delta, \tau) \in \calX$. To
    compute $R \in \bbG$, generate $(u,v) \gets \F'_{K_1}(\tau)$ and
    $(a,b) = \F'_{K_2}(\Delta)$, and output $R = g^{ua + vb}$.
\end{description}
The previous PRF is pseudo-random. The proof of the following theorem can be
found in~\cite{backes:fiore:reischuk:2013}.

\begin{theorem}\label{theo:amort-closed-form-prf}
  If $\F'$ is a PRF and the Decision Linear assumptions holds for $\calG$, then
  the function $(\tuple{\KeyGen, \F})$ presented above is a PRF.
\end{theorem}

\paragraph*{Amortized Closed-Form Efficiency for \GroupEval} The previously
defined PRF satisfies amortized closed-form efficiency for $(\GroupEval,
\vec{\mathsf{L}})$. Recall that $\function{\GroupEval}{f \times
\bbG^n}{\bbG_T}$, with arithmetic circuit $\function{f}{\bbZ^n_p}{\bbZ_p}$ and
$R_1, \dotsc, R_n \in \bbG$ random values. $\vec{\mathsf{L}}$ is a vector
$(\mathsf{L}_1, \dotsc, \mathsf{L}_n)$ with each $\mathsf{L}_i = (\Delta,
\tau_i) \in \calX$.
\begin{description}
  \item[$\CFEval^\off_{\GroupEval, \vec{\tau}}(K, f) \to \omega_f$:]
    Let $K = (\bgpp, K_1, K_2)$ be a secret key generated by
    $\KeyGen(1^\lambda)$ of a closed-form efficient PRF. Compute $\{(u_i, v_i)
      \gets \F'_{K_1}(\tau_i)\}_{i=1}^{n}$ and set $\vec{\rho_i} = (0,
      u_i, v_i)$. Basically, $\vec{\rho_i}$ are the coefficients of a degree-1
      polynomial $\rho_i(x_1, x_2)$ in two unknown variables $x_1, x_2$. Then,
      run $\PolyEval(2, f, \rho_1, \dotsc, \rho_n)$ to compute the coefficients
      $\vec{\rho}$ of a polynomial $\rho(x_1, x_2)$ such that $\forall x_1, x_2
      \in \bbZ_p \colon \rho(x_1, x_2) = f(\rho_1(x_1, x_2), \dotsc,
      \rho_n(x_1, x_2))$. Output $\omega_f = \vec{\rho}$.
  \item[$\CFEval^\on_{\GroupEval, \Delta}(K, \omega_f) \to W$:] Let $K
    = (\bgpp, K_1, K_2)$ be a secret key and $\omega_f = \vec{\rho}$ the result
    of the previous algorithm. Generate $(a,b) \gets \F'_{K_2}(\Delta)$,
    and then use the coefficients $\vec{\rho}$ to compute $w = \rho(a,b)$.
    Output $W = e(g, g)^w$.
\end{description}

\begin{theorem}
  Let $\vec{\mathsf{L}} = (\mathsf{L}_1, \dotsc, \mathsf{L}_n)$ be such that
  $\mathsf{L}_i = (\Delta, \tau_i) \in \calX$ and let $t$ be the running time
  of $\GroupEval$. Then the PRF $(\KeyGen, \F)$, extended with the algorithms
  $\CFEval^\off_{\GroupEval, \vec{\tau}}$ and $\CFEval^\on_{\GroupEval,
  \Delta}$ satisfies \emph{amortized closed-form efficiency for $(\GroupEval,
    \vec{\mathsf{L}})$} according to \refdef{def:amort-closed-form-eff-prf}, with running
    times of $\calO(t)$ and $\calO(1)$ for $\CFEval^\off_{\GroupEval,
      \vec{\tau}}$ and $\CFEval^\on_{\GroupEval, \Delta}$, respectively.
  \label{theo:amort-closed-form-group-eval}
\end{theorem}

The proof can be found in \cite{backes:fiore:reischuk:2013}.

\paragraph*{Homomorphic MAC with Efficient Verification} Having defined the PRF
that allows to more efficiently compute the $\rho$ from the verification
algorithm, we can now present the construction of an homomorphic MAC scheme
with an efficient verification algorithm.

\citescheme{backes:fiore:reischuk:2013} scheme works for circuits whose
additive gates do not get inputs labeled by constants. And just like in the
\citescheme{catalano:fiore:2013}, without loss of generality, one can use an
equivalent circuit where there is a special variable/label for the value of 1,
and the MAC of 1 can be published.  The description of the $\EVHMAC$
construction is as follows:
\begin{description}
  \item[$\KeyGen(1^\lambda) \to (\ek, \sk)$:] Run $\bgpp \gets_\$
    \calG(1^\lambda)$ to generate the description of bilinear groups with
    $\bgpp = (p, \bbG, \bbG_T, e, g)$. The message space $\calM$ is $\bbZ_p$.
    Choose a random value $\alpha \gets_\$ \bbZ_p$, and run $(K, \pp) \gets_\$
    \KeyGen_(1^\lambda)$ to obtain seed $K$ of a PRF function
    $\function{\mathsf{F}_K}{\{0,1\}^* \times \{0,1\}^*}{\bbG}$.  Output $\sk
    = (\bgpp, \pp, K, \alpha)$ and $\ek = (\bgpp, \pp)$.
  \item[$\Auth_\sk(\mathsf{L}, m) \to \sigma$:] Receives a multi-label
    $(\Delta, \tau) \in (\{0,1\}^\lambda)^2$ and a message $m \in \calM$.
    Compute $R \gets \mathsf{F}_K(\Delta, \tau)$ and set $y_0 = m$ and
    $Y_1 = (R \cdot g^{-m})^{1/\alpha}$. Output the tag $(y_0, Y_1) \in \bbZ_p
    \times \bbG$.

    If we consider the $y_1 \in \bbZ_p$ to be the unique value such that
    $Y_1 = g^{y_1}$, then $(y_0, y_1)$ are simply the coefficients of
    a degree-1 polynomial $y(x)$ such that $y(0) = m$ and $y(\alpha)
    = \phi_g^{-1}(R)$.
  \item[$\Eval_\ek(f, \vec{\sigma} \to \sigma')$:] Receives a vector of
    tags $\sigmaSpace{1}{n}$ and an arithmetic circuit
    $\function{f}{\bbZ_p^n}{\bbZ_p}$. Just as in
    \citescheme{catalano:fiore:2013} scheme, the values of $\sigma_i$'s are
    not valid messages in $\bbZ_p$, and so it is necessary to specify the
    sub-routine $\GateEval$.
    At a given gate $f_g$, given two tags $\sigma_1, \sigma_2$, $\GateEval$
    proceeds as follows:
    \begin{description}
      \item[$\GateEval_\ek(f_g, \sigma_1, \sigma_2) \to \sigma'$:] Let
        $\sigma_i = (y_0^{(i)}, Y_1^{(i)}, \hat{Y}_2^{(i)}) \in \bbZ_p \times
        \bbG \times \bbG_T$ for $i = 1,2$. Assume that $\hat{Y}_2^{(i)} = 1$
        whenever it is not defined.

        If $f_g = +$, then compute $(y_0, Y_1, \hat{Y}_2)$ as:
        \begin{itemize}
          \item $y_0 = y_0^{(1)} + y_0^{(2)}$
          \item $Y_1 = Y_1^{(1)} \cdot Y_1^{(2)}$
          \item $\hat{Y}_2 = \hat{Y}_2^{(1)} \cdot \hat{Y}_2^{(2)}$
        \end{itemize}
        If $f_g = \times$, then compute $(y_0, Y_1, \hat{Y}_2)$ as:
        \begin{itemize}
          \item $y_0 = y_0^{(1)} \cdot y_0^{(2)}$
          \item $Y_1 = (Y_1^{(1)})^{y_0^{(2)}} \cdot
            (Y_1^{(2)})^{y_0^{(1)}}$
          \item $\hat{Y}_2 = e(Y_1^{(1)}, Y_1^{(2)})$

        Since this construction assumes that $deg(f) \leq 2$, it can be assumed
        that $\sigma_i = (\tuple{y_0^{(i)}, Y_1^{(i)}}) \in \bbZ_p \times \bbG$ for
        $i = 1,2$.
        \end{itemize}
        If $f_g = \times$ and one of the two inputs is a constant $c \in
        \bbZ_p$ (assume that $\sigma_2$ is the constant), then compute $(y_0,
        Y_1, \hat{Y}_2)$ as:
        \begin{itemize}
          \item $y_0 = c \cdot y_0^{(1)}$
          \item $Y_1 = (Y_1^{(1)})^c$
          \item $\hat{Y}_2 = (\hat{Y}_2^{(1)})^c$
        \end{itemize}
    \end{description}
    Return $\sigma' = (y_0, Y_1, \hat{Y}_2)$.
  \item[$\Vrfy_\sk(m, \calP_\Delta, \sigma) \to \{\mathsf{accept},
    \mathsf{reject}\}$:] Receives a message $m \in \calM$, a multi-labeled
    program $\calP_\Delta = (f, \tauSpace{1}{n}), \Delta)$ and a tag
    $\sigma = (y_0, Y_1, \hat{Y}_2)$. For $i = 1$ to $n$, compute $R_i
    \gets \mathsf{F}_K(\Delta, \tau_i)$. Then run $W \gets
    \GroupEval(f, \setSpace{R}{1}{n}) \in \bbG_T$, and check that both:
    \begin{align*}
      m & \equals y_0\\
      W & \equals e(g,g)^{y_0} \cdot e(Y_1, g)^\alpha \cdot
      (\hat{Y}_2)^{\alpha^2}
    \end{align*}
    If both checks are satisfied, output \textsf{accept}. Output
    \textsf{reject} otherwise.
\end{description}

The previous verification algorithm is still not efficient. The description of
$\EVHMAC$ needs to be complemented with \VrfyPrep and \EffVrfy to achieve
efficient verification.
\begin{description}
  \item[$\VrfyPrep_\sk(\calP) \to \vk_\calP$:] Let $\sk = (\bgpp, \pp, K,
    \alpha)$. Receives a labeled program $\calP = (f, \tauSpace{1}{n})$.
    A concise verification key $\vk_\calP = \omega$ is computed, with $\omega
    \gets \CFEval^\off_{\GroupEval, \vec{\tau}}(K, f)$.
  \item[$\EffVrfy_{\sk, \vk_\calP}(\Delta, m, \sigma) \to \{\mathsf{accept},
    \mathsf{reject}\}$:] Let $sk = (\bgpp, \pp, K, \alpha)$ and $\vk_\calP =
    \omega$ and $\sigma = (\tuple{y_0, Y_1, \hat{Y}_2})$. First, run the online
    closed-form efficient algorithm of \F for \GroupEval to compute $W \gets
    \GroupEval^\on_{\GroupEval, \Delta}(K, \omega)$. Then, it runs the same
    checks as in \Vrfy, and returns \textsf{accept} if both checks are
    satisfied, \textsf{false} otherwise.
\end{description}

\paragraph*{Correctness} $\EVHMAC$ satisfies both authentication and evaluating
correctness. The respective proofs can be found
in~\cite{backes:fiore:reischuk:2013}.

\paragraph*{Efficient Verification} $\EVHMAC$ satisfies efficient verification.
\begin{theorem}
  If $\F$ has amortized closed-form efficiency for $(\GroupEval,
  \vec{\mathsf{L}})$, then $\EVHMAC$ satisfies efficient verification.
  \label{theo:evh-mac-eff-vrfy}
\end{theorem}
The verification preparation $\VrfyPrep$ runs in time equal to
$\CFEval^\off_{\GroupEval,\vec{\tau}}$, and the efficient verification
$\EffVrfy$ runs in time equal to $\CFEval^\on_{\GroupEval, \Delta}$. From
\reftheorem{theo:amort-closed-form-group-eval}, \VrfyPrep and \EffVrfy run in time
$\calO(\abs{f})$ and $\calO(1)$, respectively. The full proof can be found
in~\cite{backes:fiore:reischuk:2013}.

\paragraph*{Security} The security of $\EVHMAC$ is based on the following
theorem.
\begin{theorem}
  Let $\lambda$ be the security parameter, $\F$ be a PRF with security
  $\epsilon_\F$, and $\calG$ be a bilinear group generator. Then, any PPT
  adversary $\calA$ making $Q$ verification queries has at most probability
  \begin{equation}\label{eq:evh-mac-sec}
    \Pr[\mathsf{HomUF-CMA}_{\calA, \HMACML} = 1] \leq 2 \cdot \epsilon_\F
    + \frac{8Q}{p-2(Q-1)}
  \end{equation}
  of breaking the security of $\EVHMAC$.
  \label{theo:evh-mac-sec}
\end{theorem}

