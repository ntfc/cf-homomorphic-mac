%% ---------------------------------------------------------------------------
% INTRODUCTION
%% ---------------------------------------------------------------------------
\chapter{Introduction}\label{chap:intro}
A cloud computing provider allows a user to lease computing and storage
resources from remote computers that are far more powerful than the user's.
This enables her to perform operations over (some portion of) outsourced data
that is remotely stored and then only the results of these operations are
returned to the user. However, the current infrastructures of cloud computing
services do not provide any security against untrusted cloud operators, making
them unsuitable for storing sensitive information such as medical, financial or
any personal records.

A possible solution to this problem is the use of homomorphic cryptography.
This concept was first presented by \textcite{rivest:adleman:dertouzos:1978}
about 30 years ago. However, it remained an open problem until recently, when
the first truly homomorphic schemes were proposed. With these schemes it became
possible to perform computations on encrypted data such that the cloud
computing provider does not have access to the decrypted information, therefore
preserving \emph{privacy} of the outsourced data. The first \nom{FHE}{Fully
Homomorphic Encryption} scheme was proposed in a ground-breaking work by
\textcite{gentry:2009:FHE}. His scheme allows the computation of arbitrary
computations over encrypted data without being able to decrypt. Unfortunately
it is not yet practical, but several small improvements have been proposed
which could lead it to be practical in the near future.

But, and not only in the context of cloud computing, it is also important to
guarantee the \emph{authenticity} of the computations performed on the
outsourced data. FHE schemes by itself only guarantee privacy, and not
authenticity. To do so, two main goals need to be fulfilled: \emph{security}
and \emph{efficiency}. Security guarantees that the server is able to ``prove''
the correctness of the delegated computation. Efficiency means that the user is
able to efficiently check the proof by requiring far less resources than those
needed to execute the delegated computation (including both computation and
communication).

%% ---------------------------------------------------------------------------
% MOTIVATION
%% ---------------------------------------------------------------------------
\section{Motivation}
To guarantee authenticity for computations on outsourced data a new set of
homomorphic primitives were defined: homomorphic signatures (asymmetric
setting) and homomorphic message authenticators (symmetric setting).

The first notions of homomorphic signatures were introduced around 15 years ago
by \textcite{rivest:micali:chari:rabin:2001}, and then
\textcite{johnson:molnar:song:wagner:2002} presented a more complete
definition. An homomorphic signature scheme allows a user to generate
signatures $\sigmaSpace{1}{k}$ on messages $\mSpace{1}{k}$ so that later anyone
(without the private signing key) can compute a signature $\sigma'$ that is
valid for the value $m' = f(\mSpace{1}{k})$. Also, it allows anyone that holds
the public verification key to verify the results.
\textcite{boneh:freeman:2011} constructed a scheme that evaluates multivariate
polynomials on previously signed data, but they also stated that the
construction of a fully homomorphic signatures scheme remains an open problem.

The secret key analogue of homomorphic signatures are homomorphic message
authenticators (homomorphic \nom{MAC}{Message Authentication Code}s for short)
that were introduced by \textcite{gennaro:wichs:2012}. In this setting, there
is a public evaluation key and a secret key shared by Alice and Bob, which is
used to authenticate some messages $\mSpace{1}{k}$. Then, anybody who has the
public evaluation key can compute a short tag $\sigma$ that certifies the value
$y = f(\mSpace{1}{k})$ as the output of $f$. Notice that $\sigma$ does not
simply authenticate $y$ out of context, it only certifies it as the output of
a specific $f$. Later, Bob can verify that $y$ is indeed the output of
$f(\mSpace{1}{k})$ without even knowing the value of the messages.
%
\textcite{gennaro:wichs:2012} also presented a construction with support for
arbitrary computations, a fully homomorphic MAC, but its security is based on
a weaker model (the adversary cannot ask for verification queries, only for
authentication queries).

To overcome this problem, and relying on much of the work from
\citeauthor{gennaro:wichs:2012}, \textcite{catalano:fiore:2013} constructed a
very simple and efficient scheme with support for a more restricted class of
computations. Furthermore, its security is based on a stronger model with
support for multiple verification queries.
%
But both of these constructions suffered from a common problem: the
verification algorithm runs in time proportional to the size of the function.
The scheme from \textcite{backes:fiore:reischuk:2013} tries to solve this
problem, and it is the first homomorphic MAC scheme with efficient (amortized)
verification.

\begin{comment}
As for the homomorphic MAC schemes, they all rely on the theory of arithmetic
circuits for the computation of functions over some messages. Simply put, an
arithmetic circuit over a field $\bbF$ and a set of variables $X
= \{\tauSpace{1}{k}\}$, is a directed acyclic graph with the following
properties.  Each node in the graph is called \emph{gate}. Gates with in-degree
0 are called \emph{input} gates while gates with out-degree 0 are called
\emph{output} gates. Each input gate is labeled by either a \emph{variable} or
a \emph{constant}. A variable input gate is labeled with a binary string
$\tau_i$ and can take arbitrary values in $\bbF$. A constant input gate is
labeled with some constant $c$ and can only take some fixed value $c \in \bbF$.
Each \emph{internal} gate (gates with in/out-degree greater than 0) is labeled
by either $+$ or $\times$, sum and product gates, respectively. The \emph{size}
of the circuit is the number of its gates. The \emph{depth} is the length of
the longest path from input to output.

Polynomials are evaluated with arithmetic circuits as follows: input gates
compute the polynomial defined by their labels. Sum gates compute the
polynomial obtained by the sum of the (two) polynomials on their incoming
wires, while product gates compute the product of the two polynomials on their
incoming wires. The output of the circuit is the value contained on the
outgoing wire of the output gate.
\end{comment}

%% ---------------------------------------------------------------------------
% OBJECTIVES
%% ---------------------------------------------------------------------------
\section{Objectives}
We intend to study the current homomorphic MAC constructions in order to better
understand their strengths and pitfalls. Having realized what those are, we
want to implement one of these constructions, \textcite{catalano:fiore:2013} to
be more precise. The reason we ignore the implementation of
\textcite{gennaro:wichs:2012} is because of its reduced security and because it
relies heavily on lattice based techniques, which are not yet efficient.

Our main goal is then to measure the overhead introduced by the homomorphic
evaluation of \cite{catalano:fiore:2013}. In practical terms, what we intend to
measure is the amount of extra work that the external server must do to
homomorphically compute the function in comparison to a normal computation.

We will also measure the complexity of all the other operations of
\textcite{catalano:fiore:2013} in order to check if it is already practical.

\begin{comment}
In this work we will mainly focus on homomorphic MAC schemes, and all of them
rely on the theory of arithmetic circuits.
%
In practical terms, it is important to stress that arithmetic circuits should
be seen as computing \emph{specific} polynomials in $\bbF[x]$ rather than
functions $\bbF^{|x|}$ to $\bbF$. Essentially, one is interested in the formal
computation of polynomials rather than the functions that these polynomials
define. This is because a function may be expressed by a polynomial in several
ways, while, in general, a polynomial defines a unique function.\footnote{A
  good example of this case is the zero function. $f = x^2 + x$ is clearly not
the zero function, unless it is in $\bbF_2$.}

The scheme from \textcite{gennaro:wichs:2012} relies heavily on lattice based
techniques, which are not practically efficient, and given that the scheme's
security is based on a weaker security model, we choose not to study its
implementation, but only its theoretical definitions. We will instead focus our
implementation efforts on the last two
schemes~\cite{catalano:fiore:2013,backes:fiore:reischuk:2013} that are
efficient in theory, but it is not known if they have any practical
application.  Our main objective is to determine how practicable they are.

However, it is still interesting to see how both homomorphic signatures and MAC
schemes work. We will study these schemes to see what functionalities they
offer and what limitations they might have.

In the end, an overall comparison between the implemented schemes will be
presented in order to determine which one is practically more efficient.
\end{comment}

%% ---------------------------------------------------------------------------
% STRUCTURE OF THE DOCUMENT
%% ---------------------------------------------------------------------------
\section{Structure of the document}
This document is split in seven chapters. \refchapter{chap:intro} is an
introduction to the problem of homomorphic authentication and describes what is
the purpose of this dissertation.

\begin{description}
  \item[\refchapter{chap:delegation}] presents a brief review of the current
    main techniques used for delegation of computations.
  \item[\refchapter{chap:circs}] presents an alternative model of computation
    based on circuits. We start by defining what are circuits, and we explain
    how they can be represented and how can we evaluate them. We also present
    two conversion tools that convert \texttt{C} into arithmetic circuits.
  \item[\refchapter{chap:homo-auth}] describes the analysis we made to the
    homomorphic authentication primitives. We start by introducing the necessary
    preliminaries and definitions of homomorphic signatures and MACs, and then
    by presenting the actual constructions for both settings. In the end we
    state what the current limitations and open problems are in both settings.
  \item[\refchapter{chap:algebraic}] introduces the main algebraic operations
    necessary for our work. The main point of this chapter is the study of an
    efficient polynomial multiplication algorithm based on FFT.
  \item[\refchapter{chap:impl}] presents our experimental results. We show here
    how our implementation performs.
\end{description}

Finally, in \refchapter{chap:conclusion} we present the final conclusions
of all the work we developed, as well as what can be done in the future.
