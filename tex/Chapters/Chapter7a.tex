\chapter{Conclusion}\label{chap:conclusion}
We now present a comprehensive synthesis of all the work accomplished, as well
as what could be done in the future in order to further enhance this work.

Once the objectives of this dissertation were defined, we started studying some
interesting primitives regarding delegation of computations. We did an informal
review of Homomorphic Encryption, Verifiable Computation and Succinct
Non-Interactive Arguments of Knowledge, as well of Homomorphic Authentication.

All of these primitives need some way to represent the computations so that
both clients and servers are able to execute or verify them. The common way of
doing it is by using a circuit representation of the program -- boolean or
arithmetic. We also studied two tools that are able to convert code from higher
level languages to an equivalent circuit representation.

We then proceeded to make a more deep analysis of Homomorphic Authentication,
specially Homomorphic Signatures and Homomorphic MACs. We were able to
list and understand their current limitations in both security and
practicability, which lead us to the conclusion that Homomorphic Signatures are
not yet practical nor totally secure.

We then chose to implement the two constructions of
\textcite{catalano:fiore:2013} because they seemed to be ones with more
practical application. Our main focus was on determining if the homomorphic
computation of a function (to obtain a MAC) was practical so that it could
easily be computed by a more powerful external machine.

In order to achieve efficient results, we had to study an efficient method for
the multiplication of two polynomials in $\Theta(n \log{n})$ time: the Fast
Fourier Transform. This faster method becomes specially critical when the
degree of the polynomials is high.

Finally, after having an implementation of \textcite{catalano:fiore:2013} in
\texttt{C}, and by using the FFT for polynomial multiplications, we were ready
to measure its performance by using random polynomials, described as arithmetic
circuits.

As we stated in the previous chapter, the obtained results were very positive,
which means that \textcite{catalano:fiore:2013} is practical. The major problem
is within the homomorphic computation as expected, but if we consider that these
computations are supposed to be executed by a powerful machine, our results
seem very good.

\section{Future work}
From a practical point of view, there are many possible things that could be
done in the future. The first and the most obvious one is to implement the work
of \textcite{backes:fiore:reischuk:2013}, which we could use to obtain an
efficient verification algorithm (in the amortized sense).

One other area with room for improvement is the size of the input messages we
use, which now is only of 128 and 256 bits. If we supported bigger message
sizes we would be able to use more practical message values. A possible
solution would be the use of a \textcite{lubyrackoff} construction (using AES
as the round function) that would allows us to use messages up to 512 bits.

Another interesting work would be the creation of a library with support for
various Homomorphic Authentication primitives, much like the efforts done with
HELib\footnotemark for Homomorphic Encryption. Such library could also have the
possibility of a better communication with
\citetool{parno:howell:gentry:raykova:2013} or \citetool{tinyram} to easily
convert \texttt{C} programs into arithmetic circuits.
\footnotetext{\url{https://github.com/shaih/HElib}}
