\chapter{Implementation and experimental results}\label{chap:impl}
In the previous chapters we have introduced the two homomorphic MACs
(\citescheme{catalano:fiore:2013}1 and \citescheme{catalano:fiore:2013}2) that
we intend to implement, as well as the necessary theory behind them. Now we
will present some implementation details, and finally an analysis of the
efficiency of the two homomorphic MACs.

\section{Programming Language and Libraries}
Our choice of programming language was \texttt{C}. It is a language that
provides more low-level operations which are quite useful to cryptographic
applications. However, that can be both an advantage and a disadvantage, as it
is easier too create complicated situations with nasty side-effects, which are
not clear at first sight. But for our work we think that by using
\texttt{C} we can obtain the best possible results that otherwise wouldn't be
possible in some other higher level languages such as Python or Java.

For cryptographic purposes we always need to work with big numbers, but
\texttt{C} only has native support for fixed-length arithmetic (normally with
machine specific length). Therefore we need to either use an existing big
number library, or to implement our own. The latter option is clearly a bad
idea since we already have at our disposal many efficient and mature libraries
ready for a wide range of machines and architectures.

We choose to use GMP \cite{gmp} for ``bignum'' arithmetic. GMP is designed to
have good performance with both small numbers and big numbers with thousands of
bits. GMP chooses algorithms depending on the size of the operands, so the
algorithm used for a small number multiplication is different from the one used
to multiply two ``bignums''. The great speed of GMP is the fruit of many years
of optimizations provided by the community, which we can make use of without
having to worry about the underlying machine's hardware.  Besides ``bignum''
arithmetic, GMP also provides the facilities for random numbers generation.

However, GMP does not have support for polynomial arithmetic, which we need for
\citescheme{catalano:fiore:2013} tags. Once again, instead of implementing from
scratch all the algorithms presented in \refchapter{chap:algebraic}, we decided
to use an established library, in this case NTL \cite{ntl}.  NTL is
a high-performance, portable, \texttt{C++} library with support for many Number
Theory operations, including the FFT multiplication that we need\footnotemark.
\footnotetext{Although it is written in \texttt{C++}, we can still link it in
such a way that it can be used with our \texttt{C} code.}

Lastly, \citescheme{catalano:fiore:2013} makes use of a PRF in both
authentication and verifications phases. Just as suggested by
\citeauthor{catalano:fiore:2013}, we can use AES since it is extremely
efficient. We used Libgcrypt's \cite{libgcrypt} implementation of AES. Besides
cryptographic protocols, Libgcrypt also has support for big number arithmetic,
but is based on an older version of GMP, and therefore it is not as efficient
for ``bignum'' operations.

\section{Polynomial arithmetic}
We have already seen that (univariate) polynomial arithmetic is critical for
the efficiency of the \citescheme{catalano:fiore:2013} schemes.  Both addition
and multiplication by a constant take $\Theta(n)$ time, where $n$ is the number
of coefficients of a polynomial. But on the other hand, the classical
multiplication method takes $\Theta(n^2)$ time, and that's why it's better to
use the FFT multiplication method described in \refchapter{chap:algebraic},
which takes $\Theta(n \log{n})$ time.

We now present a brief overview on how we implemented these three operations
using the libraries mentioned in the previous section.

\paragraph*{Addition}
Polynomial addition is quite straightforward: given with two polynomials
in coefficient representation (i.e., simple arrays in \texttt{C}), addition is
simply the addition of two vectors element-by-element. Such addition is done
using the function from GMP.

%\input{Algorithms/poly-add}

\paragraph*{Multiply by constant}
Multiplication by constant is fairly similar to addition, but where one of the
arrays has just one element i.e., the constant value. By using the GMP
multiplication function, we can take advantage of the possible optimizations
present in it where a different algorithm (classical, Karatsuba, FFT, etc) is
chosen based on the size of the operands.

%\input{Algorithms/poly-constmul}

\paragraph*{Multiplication of two polynomials}
It is not hard to implement a machine specific FFT (i.e., using only machine
numbers such as \texttt{float}'s), but the complexity involved to achieve an
implementation closer to $\Theta(n \log{n})$ is very high. One other aspect is
that we are working with big numbers, and to implement FFT with such numbers
would introduce even more complexity.
The time spent in trying to achieve an efficient implementation would be
significant, and we probably would never end up with a time closer to $\Theta(n
\log{n})$. And also, when implementing a FFT, it is usual to use an iterative
version rather than the recursive version of \refsec{sec:fft} which improves
efficiency, but is also significantly raises the efforts of implementation.

With these limitations in mind, \textcite{Fateman:2010} studied the impact of
using libraries like NTL instead of choosing to implement FFT from scratch, and
reached the conclusion that it is much better to rely on an existent library
given that it is much simpler, and guaranteed to be fast (if the library is
obviously efficient).
%His technique works by encoding the coefficients as a single large integer,
%then use the integer multiplication GMP or NTL (for such large integers, FFT
%multiplication is used), and then decodes them using a suitable technique.
\citeauthor{Fateman:2010} also noticed that using NTL for FFT multiplication
can be more efficient if NTL is compiled with GMP support\footnotemark.
\footnotetext{NTL's author also advises to compile NTL with GMP support for
better results}

% Lixo
%For the case of multivariate polynomial multiplication, we take a simple
%approach. We start by converting the coefficients (as GMP \texttt{mpz\_t}'s)
%into NTL coefficients, represented as \texttt{ZZ\_p}'s. With them, we can
%construct a NTL polynomial \texttt{ZZ\_pX} and then call the FFT multiplication
%function of two polynomials. Finally, it is only a matter of converting the
%resulting polynomial coefficients' back to GMP coefficients.

To use the NTL's multiplication function we have to first convert the GMP
numbers to the corresponding NTL format, and then use them to create a NTL
polynomial. After the NTL multiplication, the inverse conversion is done to
obtain the resulting polynomial as GMP coefficients.  These conversions
obviously introduce a small, non-visible, overhead in the overall
multiplication process, but it's only a very small price to pay for
a full-fledged FFT multiplication function.

%\todo[inline,size=\normalsize]{BEGIN OF CONTENT UNKNOWN}
%To measure the efficiency of polynomial multiplication, we first generate two
%lists of coefficients with $n$ random values modulo some random prime $p$ of
%roughly $\lambda$ bits. Then we use both the classical and
%the efficient multiplication methods over them. The results obtained are
%summarized in below in \todo{ref}.
%
%\pgfplotstableread{Data/polymul-128.dat}{\polymulone}
\pgfplotstableread{Data/polymul-256.dat}{\polymultwo}

\begin{figure}
  \centering
  \begin{minipage}[b]{.5\linewidth}
    %%%%%%%%%%%%%%%%%%%%%%
    \begin{tikzpicture}[scale=0.9]
      \begin{axis}[
        xlabel={$n$},
        every axis x label/.style={
          at={(axis description cs:0.5,-0.05)},
          anchor=north,
        },
        ylabel={Time (ms)},
        every axis y label/.style={
          at={(axis description cs:-0.2,0.5)},
          rotate=90,
          anchor=north,
        },
        legend style={at={(0.4,0.9)}},
        axis x line=bottom,
        axis y line=left,
        style={font=\scriptsize},
        ]
        
        \addplot+[smooth,mark=none,red,line width=0.5pt] table
          [x={n}, y={classical},
          %select coords between index={1}{10}
          ]
            {\polymulone};
        \addlegendentry{$\lambda = 128$};
        
        \addplot+[smooth,mark=none,blue,line width=0.5pt] table
          [x={n}, y={classical},
          %select coords between index={1}{10}
          ]
            {\polymultwo};
        \addlegendentry{$\lambda = 256$};
      \end{axis}
    \end{tikzpicture}
    \subcaption{Classical $\Theta(n^2)$ multiplication.}\label{fig:polymul-nofft}
  \end{minipage}%
  \begin{minipage}[b]{.5\linewidth}
    %%%%%%%%%%%%%%%%%%%%%%
    \begin{tikzpicture}[scale=0.9]
      \begin{axis}[
        xlabel={$n$},
        every axis x label/.style={
          at={(axis description cs:0.5,-0.05)},
          anchor=north,
        },
        ylabel={Time (ms)},
        every axis y label/.style={
          at={(axis description cs:-0.15,0.5)},
          rotate=90,
          anchor=north,
        },
        legend style={at={(0.4,0.9)}},
        ytick={0,5,...,100},
        axis x line=bottom,
        axis y line=left,
        style={font=\scriptsize},
        ]
        
        \addplot+[smooth,mark=none,red,line width=0.5pt] table
          [x={n}, y={fft}
          %select coords between index={1}{10}
          ]
            {\polymulone};
        \addlegendentry{$\lambda = 128$};
        
        \addplot+[smooth,mark=none,blue,line width=0.5pt] table
          [x={n}, y={fft}
          %select coords between index={1}{10}
          ]
            {\polymultwo};
        \addlegendentry{$\lambda = 256$};
      \end{axis}
    \end{tikzpicture}
    \subcaption{$\Theta(n \log{n})$ multiplication with FFT.}\label{fig:polymul-fft}
  \end{minipage}%
  \caption{Times (in ms) of multiplication of two polynomials with degree $n
  = 25$ to $2500$. There is clearly a huge gain by using FFT multiplication.}
  \label{fig:polymul}
\end{figure}

%\todo[inline,size=\normalsize]{END OF CONTENT UNKNOWN}

\section{Experimental results}
We now present our experimental results by first explaining how we generated
our test case circuits, and then by providing more details of the
\citescheme{catalano:fiore:2013}'s performance.
All tests were performed using a x86\_64 GNU/Linux on a machine with 4GB RAM
and an Intel(R) Core(TM)2 Duo CPU T9600 2.8GHz. The \texttt{C} compiler was gcc
4.9.1 x86\_64 and the library versions were: GMP 6.0.0, NTL 6.2.1 and Libgcrypt
1.6.2.

\paragraph*{Circuits generation}
To measure the performance of \citescheme{catalano:fiore:2013} we tested it
with circuits computing polynomials of various degrees.
What we did was to create random polynomials using Sage \cite{sage}, and then
convert them to an appropriate arithmetic circuit representation.

%These polynomials were generated by using Sage \cite{sage} and converted to
%\texttt{C} using Python.  Finally, we used
%\citetool{parno:howell:gentry:raykova:2013} to create the corresponding
%arithmetic circuits.
%The first stage with Sage, where we generated a polynomial of degree $n$, was
%performed as follows:
%\begin{enumerate}
%  \item Choose a random number of variables $v \in {[\frac{n}{2}, 2n]}$;
%  \item Choose a random number of coefficients $c \in {[\frac{n+v}{2},
%    2(n+v)]}$;
%  \item Use Sage to generate a random polynomial of degree $n$ with $v$
%    variables and $c$ coefficients (or terms).
%\end{enumerate}
%
%After having the polynomial from Sage, the conversion to \texttt{C} was almost
%a ``one-to-one'' conversion: $5x_1 + 10x_2^3$ would become the \texttt{C} code
%\texttt{(5 * x1) + (10 * pow(x2, 3))}. This piece of code was placed in the
%body of the \texttt{outsource} function used by
%\citetool{parno:howell:gentry:raykova:2013} in order to convert it to an
%arithmetic circuit.

%We repeated this process 100 times for each degree $d \in \{10, 20, 50, 100,
%125, 150, 200, 300\}$.
We generated polynomials of degree $n$ between 10 (small enough) and 300 (big
enough). The average dimensions of the generated circuits are shown in
\reftable{tab:circuits}, and since \citetool{parno:howell:gentry:raykova:2013}
also contains some example polynomials, we test them with
\citescheme{catalano:fiore:2013} as well (we refer to them as
\citetool{parno:howell:gentry:raykova:2013}'s polynomials).

\begin{table}[htb]\footnotesize
  \centering
  \begin{minipage}[b]{.6\linewidth}
    \centering
    \begin{tabular}{|c|c|c|c|}
      \hline
      $n$  & Inputs & Coeffs   & Muls    \\\hline
      10   & 13     & 24       & 128     \\\hline
      20   & 25     & 53       & 545     \\\hline  
      50   & 66     & 129      & 3472    \\\hline  
      100  & 127    & 307      & 16178   \\\hline  
    \end{tabular}
    %
    \begin{tabular}{|c|c|c|c|c|}
      \hline
      $n$  & Inputs & Coeffs   & Muls    \\\hline
      125  & 149    & 312      & 19766   \\\hline  
      150  & 186    & 378      & 29075   \\\hline  
      200  & 269    & 532      & 55473   \\\hline  
      300  & 372    & 741      & 112195  \\\hline  
    \end{tabular}
    \subcaption{Average dimensions of the circuits computing our randomly
    generated polynomials of degree $n$.}
  \end{minipage}%
  \begin{minipage}[b]{.4\linewidth}
    \centering
    \begin{tabular}{|c|c|c|c|}
      \hline
      $n$ & Inputs & Coeffs   & Muls    \\\hline
      12  & 4      & 247      & 186     \\\hline  
      30  & 6      & 16807    & 55250   \\\hline  
      35  & 6      & 32768    & 110624  \\\hline  
      40  & 6      & 131022   & 203428  \\\hline  
      50  & 6      & 500040   & 350042  \\\hline  
    \end{tabular}%
    \subcaption{Dimensions of the example circuits of degree $n$ included with
      \citetool{parno:howell:gentry:raykova:2013}.}
  \end{minipage}%
  \captionof{table}[Arithmetic circuits' dimensions]{Dimensions of our randomly
    generated circuits and from \citetool{parno:howell:gentry:raykova:2013}'s
    polynomial examples. \citetool{parno:howell:gentry:raykova:2013}'s
    polynomials are of lower degree, but have much more coefficients and
    multiplications than ours.}\label{tab:circuits}
  %\captionof{table}[Arithmetic circuits' average dimensions.]{Average number of
  %inputs and product gates of the randomly generated circuits, where for
  %each polynomial of degree $n$ we have 100 circuits.}\label{tab:circuits} 
\end{table}


\paragraph*{\citescheme{catalano:fiore:2013} performance}
Using the polynomials we previously generated, we can now measure the
performance of \citescheme{catalano:fiore:2013}.

The results we are more interested in are the execution times of \Eval.
In a practical scenario, \Eval is executed by the external server,
which is expected to have much more computational power than the client who
executes \Vrfy.

Because \Vrfy runs in time proportional to a circuit evaluation, we know that
execution times will be similar, with only (small) overhead of \Vrfy because of
the algebraic operations it performs over the ``polynomial tag''.

We tested both sets of polynomials with \citescheme{catalano:fiore:2013}1 and
\citescheme{catalano:fiore:2013}2, but the only difference between the two is
that the latter has a slower \KeyGen (because of the necessary $D$
exponentiations) and creates succinct tags of constant size (they are simply
a group element), which is why we omitted \citescheme{catalano:fiore:2013}2
from the results.

To measure \citescheme{catalano:fiore:2013}'s performance, we used a different
set of keys for each circuit, and the input messages and labels were chosen at
random. The detailed performance results are shown in \reftable{tab:cf-perf}
(for our randomly generated polynomials) and in
\reftable{tab:cf-perf-pinocchio} (using
\citetool{parno:howell:gentry:raykova:2013}'s polynomials). For
\reftable{tab:cf-perf} we run \citescheme{catalano:fiore:2013} once in every
one of the 100 circuits of each degree, and the results presented are the
average of all these executions; for \reftable{tab:cf-perf-pinocchio} we run
\citescheme{catalano:fiore:2013} 10 times for each circuit and show the average
times for each circuit.

\begin{table}[htb]\small
\centering
{\renewcommand{\arraystretch}{0.93} %<- modify value to suit your needs
\begin{tabular}{|r|r|rrrr|r|r|}
\hline
                                           &           & \multicolumn{1}{r}{\KeyGen}& \Auth                    & \Eval                   & \Vrfy      & Circuit                   & Tag Size  \\ 
$n$                                        & $\lambda$ & \multicolumn{1}{r}{(ms)}   & \multicolumn{1}{r}{(ms)} & \multicolumn{1}{r}{(s)} & (ms)       & \multicolumn{1}{r|}{(ms)} & (kB)      \\ \hline
% n = 10
%\multicolumn{1}{|r|}{\multirow{2}{*}{10}}  & 128       & 0.37                       & 0.42                     & 40.4e-4                 & 0.64       & 0.50                      & 0.77      \\
%\multicolumn{1}{|r|}{}                     & 256       & 1.23                       & 0.63                     & 66e-4                   & 0.62       & 0.58                      & 1.5       \\ \hline
\multicolumn{1}{|r|}{\multirow{2}{*}{10}}  & 128       & 0.37                       & 0.42                     & 0.004                   & 0.64       & 0.50                      & 0.19     \\
\multicolumn{1}{|r|}{}                     & 256       & 1.23                       & 0.63                     & 0.007                   & 0.62       & 0.58                      & 0.38     \\ \hline
% n = 20
%\multicolumn{1}{|r|}{\multirow{2}{*}{20}}  & 128       & 0.41                       & 0.60                     & 24.2e-3                 & 0.38       & 0.41                      & 1.4       \\
%\multicolumn{1}{|r|}{}                     & 256       & 1.20                       & 0.47                     & 39e-3                   & 0.46       & 0.37                      & 2.8       \\ \hline
\multicolumn{1}{|r|}{\multirow{2}{*}{20}}  & 128       & 0.41                       & 0.60                     & 0.024                   & 0.38       & 0.41                      & 0.35     \\
\multicolumn{1}{|r|}{}                     & 256       & 1.20                       & 0.47                     & 0.039                   & 0.46       & 0.37                      & 0.70     \\ \hline
% n = 50
%\multicolumn{1}{|r|}{\multirow{2}{*}{50}}  & 128       & 0.39                       & 0.44                     & 30.1e-2                 & 1.17       & 1.06                      & 3.3       \\
\multicolumn{1}{|r|}{\multirow{2}{*}{50}}  & 128       & 0.39                       & 0.44                     & 0.31                    & 1.17       & 1.06                      & 0.83     \\
\multicolumn{1}{|r|}{}                     & 256       & 1.14                       & 0.44                     & 0.47                    & 1.66       & 1.51                      & 1.66     \\ \hline
% n = 100
\multicolumn{1}{|r|}{\multirow{2}{*}{100}} & 128       & 0.37                       & 0.60                     & 2.57                    & 6.32       & 5.57                      & 1.63     \\
\multicolumn{1}{|r|}{}                     & 256       & 1.16                       & 0.79                     & 4.01                    & 8.61       & 7.44                      & 3.26     \\ \hline
% n = 125
\multicolumn{1}{|r|}{\multirow{2}{*}{125}} & 128       & 0.35                       & 0.61                     & 3.79                    & 7.79       & 6.91                      & 2.03     \\
\multicolumn{1}{|r|}{}                     & 256       & 1.07                       & 0.88                     & 5.87                    & 10.26      & 8.94                      & 4.06     \\ \hline
% n = 150
\multicolumn{1}{|r|}{\multirow{2}{*}{150}} & 128       & 0.48                       & 0.71                     & 6.94                    & 11.63      & 10.31                     & 2.43     \\
\multicolumn{1}{|r|}{}                     & 256       & 1.40                       & 0.98                     & 10.46                   & 15.48      & 13.79                     & 4.86     \\ \hline
% n = 200
\multicolumn{1}{|r|}{\multirow{2}{*}{200}} & 128       & 0.38                       & 0.88                     & 17.00                   & 23.76      & 22.50                     & 3.23     \\
\multicolumn{1}{|r|}{}                     & 256       & 1.22                       & 1.32                     & 26.40                   & 30.87      & 29.08                     & 6.49     \\ \hline
% n = 300
\multicolumn{1}{|r|}{\multirow{2}{*}{300}} & 128       & 0.34                       & 1.20                     & 51.77                   & 50.69      & 49.53                     & 4.83     \\
\multicolumn{1}{|r|}{}                     & 256       & 1.21                       & 1.83                     & 79.40                   & 64.07      & 62.94                     & 9.66     \\ \hline
\end{tabular}
}
\captionof{table}[CF1 performance]{CF1 Performance. All circuits representing
  polynomials of degree $n$ are generated as mentioned before. The security
  parameter $\lambda$ is either 128 or 256 bits. The Circuit column refers to
  a circuit evaluation (as demonstrated in \refsec{sec:circeval}) where the
  inputs are the original messages used in \Auth. The Tag Size refers only to
  \citescheme{catalano:fiore:2013}1, since that for
  \citescheme{catalano:fiore:2013}2 the tag size is a group element of either
16 or 32 bytes for $\lambda$ of 128 or 256 bits,
respectively.}\label{tab:cf-perf}
\end{table}


\begin{table}[htb]\small
\centering
{\renewcommand{\arraystretch}{0.93} %<- modify value to suit your needs
\begin{tabular}{|r|r|rrrr|r|r|}
\hline
                                           &           & \multicolumn{1}{r}{\KeyGen}& \Auth                    & \Eval                   & \Vrfy      & Circuit                   & Tag Size   \\ 
$n$                                        & $\lambda$ & \multicolumn{1}{r}{(ms)}   & \multicolumn{1}{r}{(ms)} & \multicolumn{1}{r}{(s)} & (ms)       & \multicolumn{1}{r|}{(ms)} & (kB)       \\ \hline
% n = 12
\multicolumn{1}{|r|}{\multirow{2}{*}{12}}  & 128       & 0.47                       & 0.17                     & 0.007                   & 0.22       & 0.59                      & 0.22       \\
\multicolumn{1}{|r|}{}                     & 256       & 1.33                       & 0.3                      & 0.012                   & 0.27       & 0.24                      & 0.45       \\ \hline
% n = 30
\multicolumn{1}{|r|}{\multirow{2}{*}{30}}  & 128       & 0.72                       & 0.36                     & 2.89                    & 46.75      & 47.63                     & 0.51       \\
\multicolumn{1}{|r|}{}                     & 256       & 1.74                       & 0.53                     & 4.49                    & 55.38      & 52.72                     & 1.02       \\ \hline
% n = 35
\multicolumn{1}{|r|}{\multirow{2}{*}{35}}  & 128       & 0.75                       & 0.43                     & 6.34                    & 100.42     & 97.53                     & 0.59       \\
\multicolumn{1}{|r|}{}                     & 256       & 1.85                       & 0.56                     & 10.72                   & 134.26     & 128.12                    & 1.18       \\ \hline
% n = 40
\multicolumn{1}{|r|}{\multirow{2}{*}{40}}  & 128       & 0.82                       & 1.09                     & 12.81                   & 198.60     & 201.63                    & 0.67       \\
\multicolumn{1}{|r|}{}                     & 256       & 2.33                       & 0.99                     & 19.70                   & 227.37     & 229.74                    & 1.34       \\ \hline
% n = 50
\multicolumn{1}{|r|}{\multirow{2}{*}{50}}  & 128       & 1.44                       & 0.44                     & 23.92                   & 363.63     & 352.24                    & 0.83       \\
\multicolumn{1}{|r|}{}                     & 256       & 2.09                       & 0.57                     & 36.93                   & 404.12     & 399.78                    & 1.66       \\ \hline
\end{tabular}
}
\captionof{table}[CF1 performance]{CF1 Performance. Circuits computing the
  polynomials used as examples in \citetool{parno:howell:gentry:raykova:2013}
  of degree $n$. For each circuit, the \citescheme{catalano:fiore:2013} was
  executed $N = 10$ times. The security parameter $\lambda$ is either 128 or
  256 bits. The Circuit column refers to a circuit evaluation (as demonstrated
  in \refsec{sec:circeval}) where the inputs are the original messages used in
  \Auth. The Tag Size refers only to \citescheme{catalano:fiore:2013}1, since
  that for \citescheme{catalano:fiore:2013}2 the tag size is always one group
element of either 16 or 32 bytes for $\lambda$ of 128 or 256 bits,
respectively.}\label{tab:cf-perf-pinocchio}
\end{table}


Without much surprise, the authentication algorithm \Auth is extremely fast (in
less than 2ms we can authenticate about 300 messages). Also note that in
a practical scenario, the client might not outsource (and therefore
authenticate) all the messages at once, but do it incrementally instead.

Just as expected, the homomorphic evaluation \Eval is the slowest operation.
However, the obtained results are still very good for a homomorphic primitive.
In \reftable{tab:cf-perf}, to homomorphically evaluate a polynomial of $n
= 300$, it ``only'' takes about 80s. But once again, in a realistic scenario we
would be using a much more powerful machine for the homomorphic evaluation and
the results would certainly be more efficient.

However, as mentioned by \citeauthor{catalano:fiore:2013}, the tag's size of
\citescheme{catalano:fiore:2013}1 grows linearly with the degree of the
evaluated circuit, and with $n = 300$ we obtain tags of about 10kB.
By using \citescheme{catalano:fiore:2013}2 we can obtain smaller 
tags consisting of only one group $\bbG$ element, at the cost of losing
composition of programs, and by turning the \KeyGen operation slower because of
the necessary $D$ exponentiations.

It is also not a surprise to see that the \Vrfy times do not differ much from
the circuit evaluation, with the introduction of only a small overhead
responsible for the algebraic computations.
%\todo[inline,size=\small]{ok, eu e que fiquei com a ideia errada quando mostrei
%ao Dario os tempos do Vrfy, e ele disse que nao estava a espera que fossem tao
%rapidos. Nestes casos o Vrfy e rapido porque os circuitos nao sao muito
%grandes.}

Regarding the results of \citetool{parno:howell:gentry:raykova:2013}'s
polynomials in \reftable{tab:cf-perf-pinocchio}, their circuit evaluation
consists of a much larger number of gate multiplications than our randomly
generated polynomials. Because of this, their circuit evaluation takes more
time which makes both the homomorphic evaluation \Eval and \Vrfy
slower.
%\todo[inline,size=\small]{nao sei se estou a explicar bem que eu quero dizer
%que os exemplos de polys que vem com o pinocchio sao muito maiores do que os
%que eu gerei, e por isso e que os seus tempos sao mais lentos}

The full source code and the respective documentation is available online as
a Git repository\footnotemark.
\footnotetext{\url{https://bitbucket.org/ntfc/cf-homomorphic-mac/overview}}
