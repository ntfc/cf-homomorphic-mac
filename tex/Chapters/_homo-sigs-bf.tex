%% ---------------------------------------------------------------------------
% BONEH-FREEMAN CONSTRUCTION
%% ---------------------------------------------------------------------------
\subsection{Boneh-Freeman}\label{subsec:bf}
\citescheme{boneh:freeman:2011} is the first homomorphic signature scheme that
is capable of evaluating multivariate polynomials of bounded degree. All
previous schemes focused solely on \emph{linear functions}, while their scheme
allows the computation of various statistical functions like mean or standard
deviation.
%
They also proposed an improved linear scheme with support for linear functions
for small fields $\bbF_p$, specially for $p = 2$.

Here we will present an overview of their two constructions. A more detailed
description can be found in~\cite{boneh:freeman:2011}, as well as suitable
parameters and sampling algorithms necessary for the full definition of the
scheme. Both constructions rely on lattice-based techniques. They are build on
the ``hash-and-sign'' signatures of
\textcite{gentry:peikert:vaikuntanathan:2008}.

Abstractly, in~\citescheme{gentry:peikert:vaikuntanathan:2008}, the public key
is a lattice $\Lambda \subset \bbZ^\lambda$ and the secret key is a short basis
of $\Lambda$. To sign a message $m$, the holder of the secret key hashes $m$ to
an element $H(m) \in \bbZ^\lambda/\Lambda$ and samples a short vector $\sigma$
from the coset of $\Lambda$ defined by $H(m)$. The verification of $\sigma$ is
simply checking that $\sigma$ is short and that $\sigma \bmod{\Lambda} = H(m)$.

\subsubsection*{Linearly homomorphic scheme}
In a (linearly) homomorphic signature scheme, the message space is defined as
$\bbF^\lambda_p$ and a function
$\function{f}{(\bbF^\lambda_p)^n}{\bbF^\lambda_p}$ is encoded as
$f(\mSpace{1}{n}) = \sum_{i = 1}^{n}c_i m_i$. The $c_i$ are integers in $(-p/2,
p/2]$ and the description of $f$ is  $\langle f \rangle := (c_1, \dotsc, c_n)
\in \bbZ^n$.
%
To authenticate both the message and the function, as well as binding them
together, a single \citescheme{gentry:peikert:vaikuntanathan:2008} signature is
computed that is simultaneously a signature on the (unhashed) message $m \in
\bbF^\lambda_p$ and a signature on the hash of $\langle f \rangle$.

Such signature is obtained by using the ``intersection method''.
\textcite{boneh:freeman:2011} described it as follows. Let $\Lambda_1$ and
$\Lambda_2$ be $\lambda$-dimensional integer lattices with $\Lambda_1
+ \Lambda_2 = \bbZ^\lambda$. Suppose $m \in \bbZ^\lambda/\Lambda_1$ is
a message and $\function{\omega_\tau}{\bbZ^n}{\bbZ^\lambda/\Lambda_2}$ is
a hash function, dependent of $\tau$, that maps encodings of functions $f$ to
elements of $\bbZ^\lambda/\Lambda_2$.
%
Since $m$ defines a coset of $\Lambda_1$ in $\bbZ^\lambda$ and $\omega_\tau$
defines a coset of $\Lambda_2$ in $\bbZ^\lambda$, by the \nom{CRT}{Chinese
Remainder Theorem} the pair $(m, \omega_\tau(\langle f \rangle))$ defines
a unique coset of $\Lambda_1 \cap \Lambda_2$ in $\bbZ^\lambda$. Then, a short
vector $\sigma$ is computed with the property that $\sigma
= m \bmod{\Lambda_1}$ and $\sigma = \omega_\tau(\langle f \rangle)
\bmod{\Lambda_2}$. The vector $\sigma$ is a signature on $(\tau, m, \langle
f \rangle)$ that ``binds'' $m$ and $\omega_\tau(\langle f \rangle)$ -- an
attacker cannot generate a new short vector $\sigma'$ from $\sigma$ such that
$\sigma = \sigma' \bmod{\Lambda_1}$ but $\sigma \neq \sigma' \bmod{\Lambda_2}$.

In this linear scheme, $\bbF^\lambda_p \cong \bbZ^\lambda/\Lambda_1$ with the
isomorphism given explicitly by the map $\vec{\mathrm{x}} \mapsto
(\vec{\mathrm{x}} \bmod{\Lambda_1})$, and $\bbF^\ell_q \cong
\bbZ^\lambda/\Lambda_2$.

Using the above method, $\Sign_\sk(\tau, m, i)$ generates a signature on
$(\tau, m, \langle \pi_i \rangle)$, where $\pi_i$ is the $i$th \emph{projection
function} defined by $\pi_i(\mSpace{1}{n}) = m_i$ and encoded by $\langle \pi_i
\rangle = \vec{\mathrm{e}}_i$, the $i$th unit vector in $\bbZ^n$.

To authenticate the linear combination $m = \sum_{i = 1}^{n}c_i m_i$ previously
authenticated, compute the signature $\sigma := \sum_{i = 1}^{n}c_i \sigma_i$.
If $n$ and $p$ are sufficiently small, then $\sigma$ is a short vector. The
following homomorphic property is achieved:
\begin{align}
  \sigma \bmod{\Lambda_1} &= \sum_{i = 1}^n c_i m_i = m, \text{and} \nonumber \\
  \sigma \bmod{\Lambda_2} &= \sum_{i = 1}^n c_i \omega_\tau(\langle \pi_i
  \rangle) = \sum_{i=1}^{n}c_i \omega_\tau(\vec{\mathrm{e}}_i).
  \label{eq:bf-lin-homo-prop}
\end{align}

If $\omega_\tau$ is linearly homomorphic then $\sum_{i = 1}^n c_i
\omega_\tau(\vec{\mathrm{e}}_i) = \omega_\tau\left( (c_1, \dotsc, c_n \right))$
for all $c_i \in \bbZ$. Since $(c_1, \dotsc, c_n)$ is exactly the encoding
$\langle f \rangle$, the signature $\sigma$ authenticates both the message $m$
and the fact that $m$ is indeed the output of $f$ applied to the original
messages $m_1, \dotsc, m_n$.

Verification of a signature $\sigma$ is simply checking that
\begin{align}
  \sigma \bmod{\Lambda_1} & \equals m, \nonumber \\
  \sigma \bmod{\Lambda_2} & \equals \omega_\tau(\langle f \rangle).
  \label{eq:bf-lin-vrfy}
\end{align}

\paragraph*{Correctness} The above linearly homomorphic scheme satisfies both
authentication and evaluation correctness. The proofs can be found
in~\cite{boneh:freeman:2011}.

\paragraph*{Unforgeability} The security is based on the \nomsf{SIS}{Small
Integer Solution} problem on $q$-ary lattices for some prime $q$. More
precisely, for security parameter $\lambda$ it follows:
\begin{theorem}\label{theo:bf-lin-uf}
  If $\mathsf{SIS}_{q, \lambda, \beta}$ is infeasible for a suitable $\beta$,
  then the above linearly homomorphic scheme is unforgeable in the random
  oracle model.
\end{theorem}

To prove it, it is shown that a successful forger can be used to solve the
\textsf{SIS} problem in the lattice $\Lambda_2$, which for random $q$-ary
lattices (and suitable parameters) is as hard as standard worst case lattice
problems. The full proof and the suitable parameters can be found
in~\cite{boneh:freeman:2011}.

\paragraph*{Privacy} The above linearly homomorphic signature scheme is weakly
context hiding. Based on the fact that the distribution of a linear combination
of samples from a discrete Gaussian is itself a discrete Gaussian, it is proved
that a derived signature on a linear combination $m' = \sum_{i = 1}^{n} c_i
m_i$ depends (up to negligible distance) only on $m'$ and the $c_i$, and not on
the original messages $m_i$. The full proof can be found
in~\cite{boneh:freeman:2011}.

\paragraph*{Length efficiency} The bit length of a derived signature only
depends logarithmically on the size of the data set. The proof, along with the
exact length of a derived signature, can be found in~\cite{boneh:freeman:2011}.

\subsubsection*{Polynomially homomorphic scheme}
The basic idea for this scheme is as follows: what if $\bbZ^\lambda$ has a ring
structure and lattices $\Lambda_1, \Lambda_2$ are ideals? Then the maps
$\vec{\mathrm{x}} \mapsto \vec{\mathrm{x}} \bmod{\Lambda_i}$ are ring
homomorphisms, and therefore adding or multiplying signatures corresponds to
adding or multiplying the corresponding messages or functions. Since any
polynomial can be computed by repeated additions and multiplications, by adding
this structure to the previous scheme it is possible to authenticate polynomial
functions on messages.

Let $F(x) \in \bbZ[x]$ be a monic, irreducible polynomial of degree $\lambda$
and let $R$ be the ring $\bbZ[x]/(F(x))$. $R$ is isomorphic to $\bbZ^\lambda$
and ideals in $R$ correspond to integer lattices in $\bbZ^\lambda$ under the
``coefficient embedding''. This embedding maps a polynomial in $R$ to an
element in an ideal lattice as follows: if $g(x) = g_0 + g_1 x + \cdots
+ g_{\lambda-1} x^{\lambda - 1}$, then coefficient embedding can be defined as:
$g(x) \mapsto (g_0, \dotsc, g_{\lambda-1}) \in \bbZ^\lambda$.

$\Lambda_1$ and $\Lambda_2$ are specially chosen to be degree one prime ideals
$\mathfrak{p,q} \subset R$ of norm $p, q$ respectively, with $\mathfrak{p}
= (p, x - a), \mathfrak{q} = (q, x - b)$. $R/\mathfrak{p}$ and $R/\mathfrak{q}$
are isomorphic with $\bbF_p$ and $\bbF_q$, respectively. The message space is
defined by $\bbF_p$. \Sign is now exactly as in the linearly homomorphic
scheme, and it returns a short signature $\sigma \in R$.

When considering polynomial functions on $\bbF_p[x_1, \dotsc, x_n]$, the
projection functions $\pi_i$ are exactly the linear monomials $x_i$, and any
polynomial function can be obtained by adding and multiplying monomials.  If an
ordering on all monomials of the form $x_1^{e_1} \cdots x_n^{e_n}$ is fixed,
then it is possible to encode any polynomial function as its vector of
coefficients, with the $i$-th unit vector $\vec{\mathrm{e}}_i$ representing the
linear monomials $x_i$ for $i = 1, \dotsc, n$.

The hash function $\omega_\tau$ is defined exactly as in the linear scheme: for
a function $f$ in $\bbF_p[x_1, \dotsc, x_n]$ encoded as $\langle f \rangle
= (c_1, \dotsc, c_\ell) \in \bbZ^\ell$, define a polynomial $\hat{f} \in
\bbZ[x_1, \dotsc, x_n]$ that reduces to $f \bmod{p}$. Then,
$\omega_\tau(\langle f \rangle) = \hat{f}(\alpha_1, \dotsc, \alpha_n)$ where
each $\alpha_i \in \bbF_q$ is the output of the hash function $H(\tau || i)$.

The evaluation algorithm uses the lifting of $f$ to $\hat{f}$.  Given $f$ and
the signatures $\sigmaSpace{1}{n} \in R$ on messages $m_1, \dotsc, m_n \in
\bbF_p$, the signature on $f(m_1, \dotsc, m_n)$ is given by
$\hat{f}(\sigmaSpace{1}{n})$.
%
It returns a short signature $\sigma \in R$ on triple $(\tau, m, \langle
f \rangle)$ such that $\sigma = m \bmod{\mathfrak{p}}$ and $\sigma
= \omega_\tau(\langle f \rangle) \bmod{\mathfrak{q}}$. Let $\sigma_i$ be
a signature on $(\tau, m_i, \langle f_i \rangle)$ for $i = 1, 2$. The following
homomorphic properties can be observed:
\begin{align}
  \sigma_1 + \sigma_2 & \text{ is a signature on } (\tau, m_1+m_2, \langle f_1
  + f_2 \rangle) \nonumber \\
  \sigma_1 \cdot \sigma_2 & \text{ is a signature on } (\tau, m_1 \cdot m_2, \langle f_1
  \cdot f_2 \rangle)
  \label{eq:bf-poly-homo-prop}
\end{align}

The verification is essentially the same as in the linearly homomorphic scheme.

\paragraph*{Correctness} The above scheme satisfies both authentication and
evaluation correctness properties. Proof can be found
in~\cite{boneh:freeman:2011}.

\paragraph*{Unforgeability} To prove that the above scheme is unforgeable, it
is shown that a successful forger can be used to find a solution to the
\nom{SIVP}{Shortest Independent Vectors Problem} in the lattice $\Lambda_2$,
which in this scheme is the ideal $\mathfrak{q}$.

\paragraph*{Privacy} For large $p$, this homomorphic scheme is not weakly
context hiding as defined in \refdef{def:hsig-priv}.
In~\cite{boneh:freeman:2011} is is shown exactly why the signature $\sigma$
leaks information about the original messages $m_0, m_1$.

\paragraph*{Length efficiency} The bit length of a derived signature only
depends logarithmically on the size of the data set, for polynomials of bounded
constant degree and for small coefficients. The proof, along with the exact
length of a derived signature, can be found in~\cite{boneh:freeman:2011}.

