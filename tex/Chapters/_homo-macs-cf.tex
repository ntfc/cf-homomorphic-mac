%% ---------------------------------------------------------------------------
% CATALANO-FIORE CONSTRUCTION
%% ---------------------------------------------------------------------------
\subsection{Catalano-Fiore}These constructions only support a restricted set of functions. More precisely,
they support polynomials $\{f_n\}$ over $\bbF$ which have \emph{polynomially
bounded degree}, meaning that both the number of variables and the degree of
$f_n$ are bounded by some polynomial $p(n)$. The class $\mathcal{VP}$ contains
all polynomially bounded degree families of polynomials that are defined by
arithmetic circuits of polynomial size and degree.

\subsubsection*{Homomorphic MAC from OWFs}
The security of this construction relies only on a \nom{PRF}{Pseudo-Random
Function} (and thus on \nom{OWF}{One-Way Function}s). It is very simple and
efficient, and allows the homomorphic evaluation of circuits
$\function{f}{\bbZ^n_p}{\bbZ_p}$ for a prime $p$ of roughly $\lambda$ bits.

This construction is restricted to circuits whose additive gates do not get
inputs labeled by constants. Without loss of generality, and when needed, one
can use an equivalent circuit where there is a special variable/label for the
value of 1, and the MAC of 1 can be published.

The description of the scheme (which we will refer to as
\citescheme{catalano:fiore:2013}1) with security parameter $\lambda$ is as
follows:
\begin{description}
  \item[$\KeyGen(1^\lambda) \to (\ek, \sk)$.] Let $p$ be a prime of
    roughly $\lambda$ bits. Choose a seed $K \gets_\$ \{0, 1\}^\lambda$ of
    a PRF function $ \function{F_K}{\{0, 1\}^*}{\bbZ_p} $ and a random value
    $\alpha \gets_\$ \bbZ_p$. Output $\ek = p$ and $\sk = (K, \alpha)$.
    The message space $\calM$ is defined as $\bbZ_p$.
  \item[$\Auth_\sk(\tau, m) \to \sigma$.] To authenticate a message $m
    \in \bbZ_p$ with label $\tau \in \{0, 1\}^\lambda$, compute $r_\tau
    = F_K(\tau)$, set $y_0 = m$, $y_1 = (r_\tau - m)/\alpha \bmod{p}$ and output
    $\sigma = (y_0, y_1)$. $y_0, y_1$ are the coefficients of a degree-1
    polynomial $y(x)$, where $y(0)$ evaluates to $m$ and $y(\alpha)$ evaluates to
    $r_\tau$.
    
    In this construction, tags $\sigma$ will be interpreted as polynomials
    $y \in \bbZ_p[x]$ of degree $d \geq 1$ in some (unknown) variable
    $x$, i.e., $y(x) = \sum_i{y_i x^i}$ (the value of $d$ is increased by
    successive calls to \Eval).
  \item[$\Eval_\ek(f, \vec{\sigma}) \to \sigma'$.] Receives an
    arithmetic circuit $\function{f}{\bbZ^n_p}{\bbZ_p}$ and a vector of tags
    $\vec{\sigma} = (\sigmaSpace{1}{n})$. Basically, \Eval consists in
    evaluating the circuit $f$ on the tags $\vec{\sigma}$ instead of evaluating
    it on messages. But since the values of $\sigma_i$'s are not valid messages
    in $\bbZ_p$, but rather polynomials in $\bbZ_p[x]$, it is necessary to
    specify how the evaluation should be done.

    \Eval proceeds gate-by-gate, and at each gate $g$, given two tags
    $\sigma_1, \sigma_2$ (or a tag $\sigma_1$ and a constant $c \in \bbZ_p$),
    it runs the sub-routine $\sigma' \gets \GateEval_\ek(g, \sigma_1,
    \sigma_2)$ that returns a new tag $\sigma'$, which is then passed as input
    to the next gate in the circuit.  When the last gate of the circuit is
    reached, \Eval outputs the tag vector $\sigma'$ obtained by running
    \GateEval on the last gate.  \GateEval is described as follows:
    \begin{description}
      \item[$\GateEval_\ek(g, \sigma_1, \sigma_2) \to \sigma'$:] Let
        $\sigma_i = \vec{y}^{(i)} = (\setSpace{y^{(i)}}{0}{d_i})$ for $i = 1,2$
        and $d_i \geq 1$.

        If $g = +$, then:
        \begin{itemize}
          \item let $d = \max(d_1, d_2)$. It is assumed that $d_1 \geq d_2$, so
            $d = d_1$.
          \item Compute the coefficients $(\setSpace{y}{0}{d})$ of the
            polynomial $y(x) = y^{(1)}(x) + y^{(2)}(x)$. This can be done
            efficiently by adding the two vectors of coefficients such that
            $\vec{y} = \vec{y}^{(1)} + \vec{y}^{(2)}$ ($\vec{y}^{(2)}$ is
            eventually padded with zeros in positions $d_1 \dotsc d_2$).
        \end{itemize}
        If $g = \times$, then:
        \begin{itemize}
          \item let $d = d_1 + d_2$.
          \item Compute the coefficients $(\setSpace{y}{0}{d})$ of the
            polynomial $y(x) = y^{(1)}(x) * y^{(2)}(x)$ using the convolution
            operator $*$, i.e., $\forall j = \interval{0}{d} : y_j
            = \sum_{i=0}^{j}{y_i^{(1)} \cdot y_{j-i}^{(2)}}$.
        \end{itemize}
        If $g = \times$ and one of the inputs\footnote{Usually, $\sigma_2$ is the
        constant.} is a constant $c \in \bbZ_p$, then:
        \begin{itemize}
          \item let $d = d_1$.
          \item Compute the coefficients $(\setSpace{y}{0}{d})$ of the
            polynomials $y(x) = c \cdot y^{(1)}(x)$.
        \end{itemize}
        Finally, \GateEval returns $\sigma' = (\setSpace{y}{0}{d})$. The size
        of a tag grows only after the evaluation of a $\times$ gate (where both
        inputs are not constants). After the evaluation of a circuit $f$, it
        holds that $|\sigma'| = \deg(f) + 1$.
    \end{description}
    
  \item[$\Vrfy_\sk(m, \calP, \sigma) \to \{\mathsf{accept, reject}\}$.]
    Let $\calP = f(\tauSpace{1}{n})$ be a labeled program , $m \in \bbZ_p$ and
    $\sigma = (\setSpace{y}{0}{d})$ be a tag for some $d \geq 1$.  Verification
    is done as follows:
    \begin{itemize}
      \item If $y_0 \neq m$, then output \textsf{reject}. Otherwise, continue.
      \item For every input wire of $f$ with label $\tau$ compute $r_\tau
        = F_K(\tau)$. Then compute $\rho = f(\setSpace{r}{\tau_1}{\tau_n})$,
        and use $\alpha$ to check if
        \begin{equation}\label{eq:cf-1-vrfy}
          \rho \equals \sum_{i=0}^{d}{y_i \alpha^i}
        \end{equation}
        If this is true, then output \textsf{accept}. Otherwise, output
        \textsf{reject}\footnote{If using an identity program $\calI_\tau$, the
        \Vrfy just checks that $r_\tau = y_0 + y_1 \cdot \alpha$ and $y_0
        = m$.}.
    \end{itemize}
\end{description}

\paragraph*{Efficiency} The generation of a tag with \Auth is extremely
efficient as it only requires one PRF evaluation (e.g., one AES evaluation).

\Eval's complexity mainly depends on the cost of evaluating the circuit $f$,
and the additional overhead due the \GateEval sub-routine and to that the tag's
size grows with the degree of the circuit. With a circuit of degree $d$, in the
worst case, this overhead is going to be $\calO(d)$ for addition gates, and
$\calO(d \log{d})$ for multiplication gates\footnote{Algorithms based on
  \nom{FFT}{Fast Fourier Transform} can be used to efficiently compute the
  convolution. See \refchapter{chap:algebraic} for details.}

The cost of \Vrfy is basically the cost of computing $\rho$ plus the cost of
computing $\sum_{i=0}^{d}{y_i \alpha^i}$, or $\calO(|f| + d)$.

\paragraph*{Correctness} Essentially, correctness follows from the special
property of the tags generated by \Auth, i.e., that $y(0) = m$ and $y(\alpha)
= r_\tau$. In particular, this property is preserved when evaluating the
circuit $f$ over tags $\sigma_1, \dotsc, \sigma_n$.

\paragraph*{Security} The security of this scheme is based on the following
theorem.
\begin{theorem}
  If $F$ is a PRF, then the previously defined homomorphic MAC scheme is secure.
  \label{first-scheme-sec}
\end{theorem}

\subsubsection*{Compact Homomorphic MAC}
In \citescheme{catalano:fiore:2013}1, the tags' size grows linearly with the
degree of the evaluated circuit. This may be acceptable in some cases, e.g.,
circuits evaluating constant-degree polynomials, but it may become impractical
in other situations, e.g., when the degree is greater that the input size of
the circuit.

To overcome this problem, a second scheme (\citescheme{catalano:fiore:2013}2)
was proposed by \citeauthor{catalano:fiore:2013} that is almost as efficient as
the previous one. An interesting property of this second scheme is that the
tags are now of constant size. But to achieve this a few restrictions arise.
First, there is a fixed a-priori bound $D$ on the degree of allowed circuits.
Second, the homomorphic evaluation has to be done in a ``single shot'' i.e.,
the authentication tags obtained from \Eval cannot be used again to compose
with other tags.

Because of these restrictions, another interesting property, \emph{local
composition}, is achieved. With this, one can keep a locally non-succinct
version of the tag that allows for arbitrary composition. Later, when it comes
to send an authentication tag to the verifier, one can securely compress such
non-succinct tag in a compact one of constant-size.

Besides a PRF, the security of this scheme relies on a re-writing of the
\emph{$\ell$-Diffie-Hellman Inversion} problem. Basically, the definition of
$\ell$-Diffie-Hellman Inversion is that one cannot compute $g^{x^{-1}}$ given
$g, g^\alpha, \dotsc, g^{\alpha^D}$.  The re-write for this scheme states that
one cannot compute $g$ given values $g^\alpha, \dotsc, g^{\alpha^D}$.

The description of the scheme (which we refer to as
\citescheme{catalano:fiore:2013}2) with security parameter $\lambda$ is as
follows:
\begin{description}
  \item[$\KeyGen(1^\lambda, D) \to (\ek, \sk)$: ] Let $D
    = \poly(\lambda)$ be an upper-bound so that the scheme supports circuits of
    degree at most $D$. Generate a group $\bbG$ of order $p$ where $p$ is
    a prime of roughly $\lambda$ bits, and choose a random generator $g
    \gets_\$ \bbG$. Choose a seed $K$ of a PRF $\function{F_K}{\{0,
    1\}^*}{\bbZ_p} $ and a random value $\alpha \gets_\$ \bbZ_p$. For $i
    = \interval{1}{D}$ compute $h_i = g^{\alpha^i}$. Output $\ek
    = (\setSpace{h}{1}{D})$ and $\sk = (K, g, \alpha)$. The message space is
    defined as $\bbZ_p$.
  \item[$\Auth_\sk(\tau, m) \to \sigma$: ] This algorithm is exactly
    the same as the one from the previous construction. Returns a tag $(y_1,
    y_2) \in \bbZ_p^2$.
  \item[$\Eval_\ek(f, \vec{\sigma}) \to \sigma'$: ] Takes as input an
    arithmetic circuit $\function{f}{\bbZ^n_p}{\bbZ_p}$ and a vector of tags
    $\vec{\sigma} = (\sigmaSpace{1}{n})$ such that each $\sigma_i \in \bbZ_p^2$
    (i.e., it is a constant-sized tag for a degree-1 polynomial).  First,
    proceed exactly as in the first scheme to obtain the coefficients
    $(\setSpace{y}{0}{d})$. If $d = 1$ then return $\sigma = (y_0, y_1)$.
    Otherwise, compute $\Lambda = \prod_{i=1}^d{h_i ^{y_i}}$ and return $\sigma
    = \Lambda$.
  \item[$\Vrfy_\sk(m, \calP, \sigma) \to \{\mathsf{accept},
    \mathsf{reject}\}$: ] Let $\calP = (f, \tauSpace{1}{n})$ be a labeled
    program, $m \in \bbZ_p$ and $\sigma$ be a tag of either the form $(y_0,
    y_1) \in \bbZ_p^2$ or $\Lambda \in \bbG$.  First, proceed as in the first
    scheme to compute $\rho$ (i.e., evaluate the circuit with inputs
    $(\setSpace{r}{\tau_1}{\tau_n})$). If $\calP$ computes a polynomial of
    degree 1, then proceed exactly as in the first scheme. Otherwise, use $g$
    to check whether the following holds:
    \begin{equation}\label{eq:cf-2-vrfy}
      g^\rho \equals g^m \cdot \Lambda
    \end{equation}
    If everything is satisfied, then output \textsf{accept}. Otherwise, output
    \textsf{reject}.
\end{description}

\paragraph*{Efficiency} This scheme is practically as efficient as
\citescheme{catalano:fiore:2013}1. Both the tagging and verification algorithms
have exactly the same complexity. In the evaluation, the complexity is still
dominated by the cost of evaluating the circuit and the cost of the sub-routine
\GateEval.

\paragraph*{Correctness} Correctness is achieved much like as in
\citescheme{catalano:fiore:2013}1. Additionally, the
\refequation{eq:cf-2-vrfy} is essentially the same as checking that $\rho
\equals \sum_{i=0}^d y_i \alpha^i$, which is the same as
\refequation{eq:cf-1-vrfy}.

\paragraph*{Security} The security of this scheme is based on the following
theorem.
\begin{theorem}
  If $F$ is a PRF and the $(D-1)$-Diffie Hellman Inversion Assumption holds in
  $\bbG$, then the previously defined homomorphic MAC scheme is secure.
  \label{theo:cf-2-sec}
\end{theorem}

\paragraph*{Local composition} As it was already mentioned,
\citescheme{catalano:fiore:2013}2 introduces local composition. This allows us
to locally keep the large version of the tag (the polynomial $y$ with its $d
+ 1$ coefficients), but the compact version $\Lambda$ is the one that is sent
to the verifier. By keeping this large $y$, it is possible to have composition
as in the previous scheme. This is particularly interesting for applications
where not too many parties are involved, as it allows for succinct tags and
local composition of partial computations at the same time.
